\documentclass{article}

\usepackage[utf8]{inputenc}
\usepackage[T1]{fontenc}
\usepackage[portuguese]{babel}
\usepackage{amsmath, amsthm, amssymb}
\usepackage{graphicx}
\usepackage{bbm}

\author{Esdras R. Carmo - 170656}
\title{Teorema de Stokes e do Divergente}
\date{\today}

\newcommand{\REAL}{\mathbbm{R}}

% Comando para integrais duplas e triplas
\newcommand{\doubleint}[2] {\iint\limits_{#1} #2}
\newcommand{\tripleint}[2] {\iiint\limits_{#1} #2}

% Comando para a norma
\newcommand{\norm}[1] {\left.\parallel #1 \right.\parallel}
\newcommand{\PartialDer}[2] {\frac{\partial #1}{\partial #2}}

% Comando para os campos vetoriais
\newcommand{\FVett}[3] {\left(#1\right) \textbf{i} + \left(#2\right) \textbf{j} + \left(#3\right) \textbf {k}}
\newcommand{\FVetd}[2] {\left(#1\right) \textbf{i} + \left(#2\right) \textbf{j}}

% Rotacional e Divergente
\newcommand{\Rot}[0] {\textit{rot }}
\newcommand{\Div}[0] {\textit{div }}

% theorems
\newtheorem{theorem}{Teorema}[section]
\newtheorem{example}{Exemplo}[section]
\newtheorem{definition}{Definição}[section]


\begin{document}
    \maketitle

    \section{Teorema de Stokes}
        Tendo uma superfície $S$ orientada, lisa por partes, cuja fronteira é formada por uma curva $C$
        fechada, simples, lisa por partes, com orientação positiva. Seja $\vec{F}$ um campo vetorial
        cujas componentes possuem derivadas parciais contínuas em região aberta de $\REAL^3$ que contém
        $S$, então tem-se:
        \[
            \oint_C \vec{F} \cdot d\vec{r} = \doubleint{S}{\Rot \vec{F} \cdot d\vec{S}}
        \]

        \subsection{Consequências}
            Se temos duas superfícies com a mesma fronteira e ambas satisfazendo as hipóteses do Teorema,
            então a integral sobre suas superfícies do rotacional de $\vec{F}$ são iguais.

            Ainda vimos que se $\Rot \vec{F} = 0$, então o campo é conservativo e a integral de linha
            sobre um caminho fechado $C$ é zero. Isso é verificado também aplicando o Teorema de Stokes.

        \subsection{Exemplos}
            \begin{example}
                Calcule $\oint_C \vec{F} \cdot d\vec{r}$, em que
                \[
                    \vec{F}(x,y,z) = (-y^2, x, z^2)
                \]
                e $C$ é a curva de intersecção do plano $y + z = 2$ com o cilindro $x^2 + y^2 = 1$
                (com orientação no sentido anti-horário quando vista por cima).

                Tomamos como superfície o plano em que intersecta o cilindro e está contido dentro dele.

                Calculando o rotacional de $\vec{F}$, temos:
                \begin{align*}
                    \Rot \vec{F} &= \vec{\nabla} \wedge \vec{F}\\
                    \Rot \vec{F} &= (0, 0, 1 + 2y)\\ 
                \end{align*}

                Então precisamos calcular $\doubleint{S}{(1 + 2y)\vec{k} \cdot d\vec{S}}$.

                Em coordenadas cilindricas, podemos descrever a superfície:
                \begin{align*}
                    x &= r\cos\theta\\
                    y &= r\sin\theta\\
                    z &= 2 - y = 2 - r\sin\theta\\
                    1 &\leq r \leq 1\\
                    0 &\leq \theta \leq 2\pi\\
                    s(r, \theta) &= (r\cos\theta, r\sin\theta, 2 - r\sin\theta)
                \end{align*}

                Agora, para calcular a integral de superfície, precisamos:
                \begin{align*}
                    s_r &= (\cos\theta, \sin\theta, -\sin\theta)\\
                    s_\theta &= (-r\sin\theta, r\cos\theta, r\cos\theta)\\
                    s_r \wedge s_\theta &= (0, r, r)\\
                \end{align*}

                Note que o produto vetorial retornou um vetor sempre acima à superfície, pois temos as componentes
                $y$ e $z$ positiva. Dessa forma, caminhando pela curva com a cabeça em sentido ao vetor normal,
                temos a superfície à nossa esquerda. Logo essa superfície tem orientação positiva.

                Portanto, calculamos a integral $I$:
                \begin{align*}
                    I &= \oint_C \vec{F} \cdot d\vec{r}\\
                    I &= \doubleint{S}{(1 + 2r\sin\theta)\vec{k} \cdot (s_r \wedge s_\theta) dA}\\
                    I &= \int_0^{2\pi} \int_0^1 (0, 0, 1 + 2r\sin\theta) \cdot (0, r, r) dr d\theta\\
                    I &= \int_0^{2\pi} \int_0^1 (r + 2r^2\sin\theta) dr d\theta\\
                    I &= \int_0^{2\pi} \left[\frac{r^2}{2} + \frac{2r^3}{3} \sin\theta \right]_0^1 d\theta\\
                    I &= \int_0^{2\pi} \left( \frac{1}{2} + \frac{2}{3} \sin\theta \right) d\theta\\
                    I &= \pi\\
                \end{align*}
            \end{example}
           
            \begin{example}
                Calcule a integral $\doubleint{S} \Rot \vec{F} \cdot d\vec{S}$ em que
                \[
                    \vec{F} = (xz, yz, xy)
                \]
                e $S$ é a parte da esfera $x^2 + y^2 + z^2 = 4$ que está dentro do cilindro $x^2 + y^2 = 1$
                e acima do plano $xy$.

                Sabemos, pelo teorema de Stokes que podemos calcular a integral $\oint_C \vec{F} \cdot d\vec{r}$
                onde $C$ é a curva dada pela interseção da esfera com o cilindro.

                Encontrando a interseção:
                \begin{align*}
                    x^2 + y^2 + z^2 &= 4\\
                    x^2 + y^2 &= 1\\
                    z^2 &= 3\\
                    z &= \sqrt{3} \textit{, pois } z > 0
                \end{align*}

                Agora, utilizando coordenadas cilindricas, temos a superfície:
                \begin{align*}
                    x &= \cos\theta\\
                    y &= \sin\theta\\
                    0 &\leq \theta < 2\pi
                \end{align*}

                Logo, temos a curva:
                \begin{align*}
                    r(\theta) &= (\cos\theta, \sin\theta, \sqrt{3})\\
                    r^\prime(\theta) &= (-\sin\theta, \cos\theta, 0)\\
                \end{align*}

                Agora calculamos a integral:
                \begin{align*}
                    I &= \oint_C \vec{F} \cdot d\vec{r}\\
                    I &= \int_0^{2\pi} \vec{F}(\vec{r}(\theta)) \cdot r^\prime(\theta) d\theta\\
                    I &= \int_0^{2\pi} (\sqrt{3}\cos\theta, \sqrt{3}\sin\theta, \cos\theta\sin\theta) \cdot
                         (-\sin\theta, \cos\theta, 0) d\theta\\
                    I &= 0
                \end{align*}
            \end{example}

    \section{Teorema do Divergente}
        Também conhecido como Teorema de Gauss. Estabelece uma relação entre a integral do divergente de um
        campo vetorial $\vec{F}$ sobre uma região com a integral de $\vec{F}$ sobre a fronteira da
        região.

        Se $E$ uma região sólida simples e seja $S$ a superfície fronteira de $E$, orientada positivamente.
        Seja $\vec{F}$ um campo vetorial com derivadas parciais contínuas, tem-se:
        \[
            \doubleint{S}{\vec{F} \cdot d\vec{S}} = \tripleint{E}{\Div \vec{F} dV}
        \]

        \subsection{Exemplos}
            \begin{example}
                Determine o fluxo do campo vetorial $\vec{F} = (z, y, x)$ sobre a esfera
                unitária $x^2 + y^2 + z^2 = 1$.

                Temos, pelo Teorema de Gauss, que:
                \[
                    I = \doubleint{S}{\vec{F} \cdot d\vec{S}} = \tripleint{E}{\Div \vec{F} dV}
                \]

                Calculamos o divergente:
                \begin{align*}
                    \Div \vec{F} &= \nabla \cdot \vec{F}\\
                    \Div \vec{F} &= 1\\
                \end{align*}

                Logo, temos a integral:
                \begin{align*}
                    I &= \tripleint{E}{dV}\\
                \end{align*}

                Como essa integral retorna o volume da esfera $E$ com raio $1$, temos:
                \[
                    I = \frac{4}{3}\pi
                \]
            \end{example}

            \begin{example}
                Calcule
                \[
                    I = \doubleint{S}{\vec{F}\cdot d\vec{S}}
                \]
                em que:
                \[
                    \vec{F} = (xy, y^2 + e^{xz^2}, \sin (xy))
                \]
                e $S$ é a superfície da região $E$ limitada pelo cilindro parabólico
                $z = 1 - x^2$ e pelos planos $z = 0$, $y = 0$, $y + z = 2$.

                Pelo teorema do divergente, temos $I$ como
                \[
                    I = \tripleint{E}{\Div \vec{F} dV}
                \]

                Calculando o divergente:
                \begin{align*}
                    \Div \vec{F} &= y + 2y = 3y
                \end{align*}

                Logo, temos a integral
                \begin{align*}
                    I &= \tripleint{E}{3y dV}\\
                    -1 &\leq x \leq 1\\
                    0 &\leq z \leq 1 - x^2\\
                    0 &\leq y \leq 2 - z\\
                    I &= \int_{-1}^1 \int_0^{1 - x^2} \int_0^{2 - z} 3y dy dz dx\\
                    I &= \frac{184}{35}
                \end{align*}
            \end{example}
\end{document}
