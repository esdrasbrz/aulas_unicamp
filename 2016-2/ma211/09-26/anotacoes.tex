\documentclass{article}

\usepackage[utf8]{inputenc}
\usepackage[T1]{fontenc}
\usepackage[portuguese]{babel}
\usepackage{amsmath}
\usepackage{bbm}

\author{Esdras R. Carmo - 170656}
\title{Multiplicadores de Lagrange}
\date{\today}

% Comando para derivadas parciais
\newcommand{\PartialDer}[2] {\frac{\partial #1} {\partial #2}}
% Comando para a norma
\newcommand{\norma}[1] {\left.\parallel #1 \right.\parallel}

\begin{document}
    \maketitle
    
    \section{Método dos Multiplicadores de Lagrange}
        \paragraph{}
        Resoluções de problemas de otimização com restrições. Dessa forma, se temos $m$ restrições $g_i(x)$, então
        existem escalares $\lambda_i$ tais que

        \begin{align*}
            \nabla f = \sum_{i=1}^m \lambda_i g_i(x)
        \end{align*}

        \paragraph{}
        Assim precisamos determinar o $x$ e os $\lambda_i$ tais que:

        \begin{align*}
            \begin{cases}
                \nabla f(x) &= \sum_{i=1}^m \lambda_i g_i(x)\\
                g_i(x) &= 0
            \end{cases}
        \end{align*}

        \paragraph{}
        Utilizando os multiplicadores de Lagrande, não faz sentido fazer o teste da matriz Hessiana pois $\nabla f(x) \neq 0$. Além disso,
        esse método sempre dará os pontos máximos e mínimos, nunca um ponto de sela.


        \subsection{Exemplo 1}
            \paragraph{}
            Precisamos de $x$, $y$, $z$ que fornecem o volume máximo de uma caixa de $12m^2$ de papelão sem
            a tampa

            \paragraph{}
            Precisamos maximizar $f(x, y, z) = xyz$ com a restrição $g(x, y, z) = xy + 2xz + 2yz - 12 = 0$, que é
            a quantidade de papelão dada no enunciado. Por lagrange, temos:

            \begin{align*}
                \begin{cases}
                    \nabla f &= \lambda \nabla g\\
                    g(x, y, z) &= 0
                \end{cases}\\
                \nabla f(x, y, z) &= (yz, xz, xy)\\
                \nabla g(x, y, z) &= (y + 2z, x + 2z, 2y + 2x)\\
            \end{align*}
            \begin{align*}
                \begin{cases}
                    yz &= \lambda (y + 2z)\\
                    xz &= \lambda (x + 2z)\\
                    xy &= \lambda (2y + 2x)\\
                    xy + 2yz + 2xz &= 12
                \end{cases}
            \end{align*}

            \paragraph{}
            Resolvendo o sistema não linear, temos, pela primeira e segunda equação:

            \begin{align*}
                \frac{yz} {y + 2z} &= \frac{xz} {x + 2z} \textit{ , }z \neq 0\\
                y(x + 2z) &= x(y + 2z)\\
                2yz &= 2xz \Rightarrow x = y\\
            \end{align*}

            \paragraph{}
            Pelas equações 2 e 3, temos:

            \begin{align*}
                \frac{xz}{x + 2z} &= \frac{xy} {2 (x + y)} \textit{ , }x \neq 0\\
                2z(x + y) &= y(x + 2z)\\
                2zx + 2zy &= xy + 2zy\\
                y &= 2z\\
            \end{align*}

            \paragraph{}
            Logo, $x = y = 2z$. Portanto, substituindo na equação 4, temos:

            \begin{align*}
                x^2 + x^2 + x^2 &= 12\\
                x &= 2 \textit{ , como }x > 0\\
            \end{align*}

            \paragraph{}
            Assim temos $(x, y, z) = (2, 2, 1)$.

        \subsection{Exemplo 2}
            \paragraph{}
            Determine os valores extremos de $f(x, y)$ no disco $x^2 + y^2 \leq 1$.

            \begin{align*}
                f(x, y) &= x^2 + 2y^2\\
            \end{align*}

            \paragraph{}
            Encontrando os pontos críticos de $f$.

            \begin{align*}
                \nabla f &= (2x, 4y) = (0, 0)\\
                x &= 0\\
                y &= 0\\
            \end{align*}

            \paragraph{}
            Portanto, $(0, 0)$ é um ponto crítico de $f$. Note que $f(0, 0) = 0$, e
            $(0, 0)$ está no interior do disco. Agora então deveremos encontrar os pontos
            da fronteira do disco.
                
            \paragraph{}
            Considere maximizar ou minimizar $f(x, y)$ sujeito à $g(x, y) = x^2 + y^2 - 1 = 0$.
            Note que $y^2 = 1 - x^2$, logo, substituindo em $f$, temos:
            
            \begin{align*}
                \varphi(x) &= -x^2 + 2\\
                \frac{d\varphi}{dx} (x) &= -2x = 0 \Rightarrow x = 0
            \end{align*}

            \paragraph{}
            Então $\varphi$ tem máximo em $x = 0$. Nesse caso, temos $y = \pm 1$ e $f(0, \pm 1) = 2$.
            Dessa forma $f$ tem valor máximo em $(0, 1)$ e $(0, -1)$ e mínimo em $(0, 0)$.

            \paragraph{}
            Utilizando o método dos multiplicadores de Lagrange, temos:
            
            \begin{align*}
                \nabla f &= (2x, 4y) \textit{ e } \nabla g = (2x, 2y)\\
            \end{align*}
            \begin{align*}
                \begin{cases}
                    2x &= \lambda 2x\\
                    4y &= \lambda 2y\\
                    x^2 + y^2 &= 1\\
                \end{cases}
            \end{align*}

            \paragraph{}
            Resolvendo o sistema, obtemos as soluções: $(\pm 1, 0)$ e $(0, \pm 1)$.
            Note que $f(\pm 1, 0) = 1$ e $f(0, \pm 1) = 2$. Então o máximo é $2$ e o mínimo $0$ (que 
            se encontra no interior da região).

        \subsection{Exemplo 3}
            \paragraph{}
            Determine o valor máximo da função $f(x, y, z) = x + 2y + 3z$ na curva da intersecção do plano
            $x - y + z = 1$ com o cilindro $x^2 + y^2 = 1$.
            \paragraph{}
            Assim precisamos maximizar a função $f$ sendo que $(x, y, z)$ satisfaz $g_1(x, y, z) = x - y + z - 1 = 0$ e $g_2(x, y, z) = x^2 + y^2 - 1 = 0$.

            \paragraph{}
            Pelo método dos Multiplicadores de Lagrange, temos:
            
            \begin{align*}
                \begin{cases}
                    \nabla f &= \lambda \nabla g_1 + \gamma \nabla g_2\\
                    g_1(x, y, z) &= 0\\
                    g_2(x, y, z) &= 0\\
                \end{cases}\\
                \begin{cases}
                    1 &= \lambda + 2\gamma x\\
                    2 &= -\lambda + 2\gamma y\\
                    3 &= \lambda\\
                    x - y + z &= 1\\
                    x^2 + y^2 &= 1\\
                \end{cases}\\
                \begin{cases}
                    x &= -\frac{1}{\gamma}\\
                    y &= \frac{5}{2\gamma}\\
                \end{cases}\\
                \text{De (5), temos}\\
                \gamma &= \pm \frac{\sqrt{29}}{2}\\
            \end{align*}

            \textbf{TERMINAR}
\end{document}
