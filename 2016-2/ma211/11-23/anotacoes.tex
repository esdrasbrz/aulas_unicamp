\documentclass{article}

\usepackage[utf8]{inputenc}
\usepackage[T1]{fontenc}
\usepackage[portuguese]{babel}
\usepackage{amsmath, amsthm, amssymb}
\usepackage{graphicx}
\usepackage{bbm}

\author{Esdras R. Carmo - 170656}
\title{Teorema de Green}
\date{\today}

\newcommand{\REAL}{\mathbbm{R}}

% Comando para integrais duplas e triplas
\newcommand{\doubleint}[2] {\iint\limits_{#1} #2}
\newcommand{\tripleint}[2] {\iiint\limits_{#1} #2}

% Comando para a norma
\newcommand{\norm}[1] {\left.\parallel #1 \right.\parallel}
\newcommand{\PartialDer}[2] {\frac{\partial #1}{\partial #2}}

% Comando para os campos vetoriais
\newcommand{\FVett}[3] {#1 \textbf{i} + #2 \textbf{j} + #3 \textbf {k}}
\newcommand{\FVetd}[2] {#1 \textbf{i} + #2 \textbf{j}}

% theorems
\newtheorem{theorem}{Teorema}[section]
\newtheorem{example}{Exemplo}[section]
\newtheorem{definition}{Definição}[section]


\begin{document}
    \maketitle
    
    \section{Introdução}
        O teorema estabelece uma relação entre a integral de linha e uma integral dupla
        sobre a superfície delimitada pela curva fechada simples.

        \subsection{Orientação}
            Em uma região fechada, a orientação positiva significa percorrê-la no sentido anti-horário, enquanto
            negativa no sentido horário.

    \section{Teorema}
        Seja $C$ uma curva plana simples, fechada, contínua por partes, orientada positivamente e $D$ a região delimitada
        por $C$. Se $P$ e $Q$ tem derivadas parciais contínuas, então:
        \[
            \int_C Pdx + Qdy = \doubleint{D}{\left( \PartialDer{Q}{x} - \PartialDer{P}{y} \right)} dA
        \]

        \subsection{Cálculo de Áreas}
            Para o cálculo de áreas da região $D$ delimitada pela curva $C$, temos:
            \begin{align*}
                A &= \int_C x dy\\
                A &= - \int_C y dx\\
                A &= \frac{1}{2} \int_C {xdy - ydx}
            \end{align*}

    \section{Exemplos}
        \begin{example}
            Calcule
            \[
                \oint_C x^4 dx + xydy
            \]
            em que $C$ é a curva triangular constituída pelos segmentos de reta de $(0,0)$ a $(1, 0)$, de
            $(1, 0)$ a $(0, 1)$ e de $(0, 1)$ a $(0, 0)$.

            Pelo Teorema de Green, temos:
            \begin{align*}
                I &= \doubleint{D}{\left( \PartialDer{Q}{x} - \PartialDer{P}{y} \right)} dA\\
                P(x,y) &= x^4\\
                Q(x,y) &= xy\\
                I &= \doubleint{D}{y} dA\\
                I &= \int_0^1 \int_0^{1 - x} y dy dx\\
                I &= \int_0^1 \left[ \frac{y^2}{2} \right]_0^{1 - x} dx\\
                I &= \frac{1}{2} \int_0^1 (1 - 2x + x^2) dx\\
                I &= \frac{1}{2} \left[ x - x^2 + \frac{x^3}{3} \right]_0^1\\
                I &= \frac{1}{6}
            \end{align*}
        \end{example}

        \begin{example}
            Calcule
            \[
                I = \int_C (3y - e^{\sin x}) dx + (7x + \sqrt{y^4 + 1}) dy
            \]
            em que $C$ é o círculo $x^2 + y^2 = 9$.

            Pelo Teorema de Green, temos:
            \begin{align*}
                I &= \doubleint{D}{\left( \PartialDer{Q}{x} - \PartialDer{P}{y} \right) dA}\\
                I &= \doubleint{D}{(7 - 3) dA}\\
                I &= 4 \doubleint{D}{dA}\\
                I &= 4 (\pi . 9) = 36 \pi\\
            \end{align*}
        \end{example}

        \begin{example}
            Determine a área delimitada pela elipse
            \[
                \frac{x^2}{a^2} + \frac{y^2}{b^2} = 1
            \]

            Pelo Teorema de Green, considerando $C$ a elipse, podemos encontrar a área como:
            \begin{align*}
                A &= \frac{1}{2} \int_C x dy - y dx
            \end{align*}

            Parametrizando a elipse, temos
            \begin{align*}
                x &= a \cos t\\
                y &= b \sin t\\
                0 &\leq t \leq 2\pi\\
                dx &= -a \sin t dt\\
                dy &= b \cos t dt
            \end{align*}

            Assim
            \begin{align*}
                A &= \frac{1}{2} \int_0^{2\pi} a \cos t (b \cos t) dt - b \sin t (-a \sin t) dt\\
                A &= \frac{1}{2} \int_0^{2\pi} (ab \cos^2 t + ab\sin^2 t) dt\\
                A &= \frac{ab}{2} \int_0^{2\pi} 1 dt\\
                A &= ab \pi
            \end{align*}
        \end{example}

        \begin{example}
            Calcule
            \[
                \oint_C y^2 dx + 3xy dy
            \]
            em que $C$ é a fronteira da região semianular $D$ contida no semiplano superior entre os círculos
            $x^2 + y^2 = 1$ e $x^2 + y^2 = 4$.

            Pela definição de integral de linha, teríamos que parametrizar 4 curvas para realizar o cálculo. No entanto, pelo
            Teorema de Green, temos
            \begin{align*}
                I &= \doubleint{D}{\left( \PartialDer{Q}{x} - \PartialDer{P}{y} \right) dA}\\
                I &= \doubleint{D}{(3y - 2y) dA} = \doubleint{D}{y dA}\\
            \end{align*}

            Utilizando coordenadas polares, podemos descrever $D$ da seguinte forma:
            \begin{align*}
                0 &\leq \theta \leq \pi\\
                1 &\leq r \leq 2\\
                dA &= r dr d\theta
            \end{align*}

            Temos então a integral
            \begin{align*}
                I &= \int_0^\pi \int_1^2 r \sin \theta r dr d\theta\\
                I &= \int_0^\pi \sin \theta d\theta \int_1^2 r^2 dr\\
                I &= \left[ - \cos \theta \right]_0^\pi \left[ \frac{r^3}{3} \right]_1^2\\
                I &= 2 \cdot \left(\frac{8}{3} - \frac{1}{3}\right)\\
                I &= \frac{14}{3}
            \end{align*}
        \end{example}

        \begin{example}
            Se
            \[
                F(x,y) = \frac{-y\textbf{i} + x\textbf{j}}{x^2 + y^2}
            \]
            Mostre que $\int_C F \cdot dr = 2\pi$ para todo caminho fechado simples que circunde a origem.

            Temos:
            \begin{align*}
                P &= \frac{-y}{x^2 + y^2}\\
                Q &= \frac{x}{x^2 + y^2}\\
                \PartialDer{P}{y} &= \frac{y^2 - x^2}{(x^2 + y^2)^2}\\
                \PartialDer{Q}{x} &= \frac{y^2 - x^2}{(x^2 + y^2)^2}
            \end{align*}

            Note que as derivadas são iguais, mas NÃO SÃO CONTÍNUAS, portanto o campo não é conservativo pois suas derivadas
            não são contínuas na origem, onde justamente o enunciado pede.

            Portanto, considerando $C$ uma curva positiva qualquer e $C_a$ um círculo negativo contendo a origem, temos uma região
            sem a origem entre $C_a$ e $C$ com derivadas contínuas, já que não possui a origem como ponto singular. Pelo Teorema
            de Green, temos que
            \begin{align*}
                \int_C F \cdot dr - \int_{C_a} F \cdot dr &= \doubleint{D}{\left( \PartialDer{Q}{x} - \PartialDer{P}{y} \right) dA} = 0\\
                \int_C F \cdot dr &= \int_{C_a} F \cdot dr\\
            \end{align*}

            Portanto, agora podemos calcular a integral sobre $C_a$.
            \begin{align*}
                x &= a \cos t\\
                y &= a \sin t\\
                0 &\leq t \leq 2\pi\\
                I &= \int_0^{2\pi} - \frac{a \sin t}{a^2} (-a \sin t dt) + \frac{a \cos t}{a^2} (a \cos t dt)\\
                I &= \int_0^{2\pi} (\sin^2 t + \cos^2 t) dt = 2\pi
            \end{align*}
        \end{example}
\end{document}
