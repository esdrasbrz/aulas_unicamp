\documentclass{article}

\usepackage[utf8]{inputenc}
\usepackage[T1]{fontenc}
\usepackage[portuguese]{babel}
\usepackage{amsmath, amsthm, amssymb}
\usepackage{graphicx}
\usepackage{bbm}

\author{Esdras R. Carmo - 170656}
\title{Integrais Triplas}
\date{\today}

\newcommand{\REAL}{\mathbbm{R}}

% Comando para integral dupla sobre região R
\newcommand{\doubleint}[1] {\iint\limits_D #1 dA}
\newcommand{\tripleint}[1] {\iiint\limits_E #1 dV}

% theorems
\newtheorem{theorem}{Teorema}[section]
\newtheorem{example}{Exemplo}[section]
\newtheorem{definition}{Definição}[section]


\begin{document}
    \maketitle

    \section{Definição para caixa retangular}
        Sendo a caixa $E = [a, b] \times [c, d] \times [r, s]$, i. e.,
        \[
            E = \{(x,y,z) \in \REAL^3 \mid a \leq x \leq b, c \leq y \leq d, r \leq z \leq s \}
        \]

        Temos:

        \begin{align*}
            \tripleint{f(x,y,z)} &= \int_a^b \int_c^d \int_r^s f(x,y,z) dz dy dx\\
        \end{align*}

        Sendo possível alterar a ordem de integração.

    \section{Cálculo de volume}
        Dado um sólido $E$, então temos o seu volume como:
        \[ V(E) = \tripleint{} \]

    \section{Exemplos}
        \begin{example}
            Calcule $\tripleint{z}$, onde $E$ é o tetraedro delimitado por $x = 0$, 
            $y = 0$, $z = 0$, $x + y + z = 1$.

            Note pelo desenho que a região é:
            \[
                E = \{ (x,y,z) \in \REAL^3 \mid (x,y) \in D, 0 \leq z \leq 1 - x - y\}
            \]

            Assim, temos:

            \begin{align*}
                \tripleint{z} &= \doubleint{\left[ \int_0^{1-x-y} z dz \right]}\\
                &= \doubleint{\left[ \frac{z^2}{2} \right]_0^{1-x-y}}\\
                \tripleint{z} &= \frac{1}{2} \doubleint{(1 - x - y)^2}
            \end{align*}

            Agora precisamos encontrar a região $D$. Portanto:
            \[ D = \{(x,y) \in \REAL^2 \mid 0 \leq x \leq 1, 0 \leq y \leq 1 - x \} \]

            Assim, temos:

            \begin{align*}
                \frac{1}{2} \int_0^1 \int_0^{1 - x} (1 - x - y)^2 dy dx\\
                \textit{Fazendo }u(y) = 1 - x - y\textit{, temos}\\
                \frac{1}{2} \int_0^1 \int_{1-x}^0 -u^2 du dx &= \frac{1}{2} \int_0^1 \left[ - \frac{u^3}{3} \right]_{1-x}^0 dx\\
                &= \frac{1}{6} \int_0^1 (1 - x)^3 dx
            \end{align*}

            Tomando $v = 1 - x$, temos:

            \begin{align*}
                \frac{1}{6} \int_1^0 -v^3 dv &= \frac{1}{6} \int_0^1 v^3 dv = \frac{1}{24}\\
            \end{align*}
        \end{example}
\end{document}
