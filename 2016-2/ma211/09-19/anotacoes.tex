\documentclass{article}

\usepackage[utf8]{inputenc}
\usepackage[T1]{fontenc}
\usepackage[portuguese]{babel}
\usepackage{amsmath}
\usepackage{bbm}

\author{Esdras R. Carmo - 170656}
\title{Derivadas Direcionais e o Vetor Gradiente}
\date{\today}

% Comando para derivadas parciais
\newcommand{\PartialDer}[2] {\frac{\partial #1} {\partial #2}}
% Comando para a norma
\newcommand{\norm}[1] {\left.\parallel #1 \right.\parallel}

\begin{document}
    \maketitle
    
    
    \section{Derivadas Direcionais}
        \paragraph{}
        Resolução de problemas para calcular a taxa de variação de função $f(x)$, sendo $x = (x_1, x_2, \dots, x_n)$,
        no ponto $a = (a_1, a_2, \dots, a_n)$ na direção de um vetor unitário $u = (u_1, \dots, u_n)$.

        \subsection{Definição}
            \paragraph{}
            Sendo $u \in \mathbbm{R}^n$:

            \begin{align*}
                D_u f(a) = \lim_{h \to 0} \frac{f(a + hu) - f(a)} {h}
            \end{align*}

        \subsection{Derivadas Parciais}
            \paragraph{}
            Derivadas direcionais generalizam as derivadas parciais, já que se $u = (1, 0)$ temos $\PartialDer{f(x, y)}{x}$.

            \begin{align*}
                \PartialDer{f(a, b)}{x} = \lim_{h \to 0} \frac {f( (a, b) + h(1, 0) ) - f(a, b)}{h}
            \end{align*}

            \paragraph{}
            Pode-se escrever as derivadas direcionais em função das parciais.
            \paragraph{}
            Sendo:

            \begin{align*}
                g(h) &= f(a + hu)\\
                g(0) &= \lim_{h \to 0} \frac{g(h) - g(0)} {h} = D_u f(a)\\      
                \text{Pela regra da cadeia, temos:}\\
                g(0) &= \sum_{j = 1}^n \PartialDer{f}{x_j} \mid_a u_j
            \end{align*}

        \subsection{Vetor Gradiente}
            \paragraph{}
            Pode-se escrever a derivada direcional de $f$ com relação a $u$ como produto escalar do gradiente $\nabla$:

            \begin{align*}
                D_u f(x) = \sum_{j = 1}^n \PartialDer{f}{x_j} u_j = \nabla f \cdot u
            \end{align*}

        \subsection{Interpretação do Vetor Gradiente}
            \begin{align*}
                D_u f = \nabla f \cdot u = \norm{\nabla f} \norm{u} \cos{\theta} = \norm{\nabla f} \cos{\theta}
            \end{align*}

            \paragraph{}
            Para qualquer direção da derivada, iremos multiplicar o vetor gradiente pelo cosseno do ângulo. Portanto o maior valor
            possível para a derivada direcional é quando $\theta = 0$, i. e., a maior variação ocorre na direção do vetor gradiente.
            \paragraph{}
            Portanto o vetor gradiente nos diz a direção de maior variação da função $f(x)$

            \subsubsection{Em $\mathbbm{R}^2$}
                \paragraph{}
                O vetor gradiente $\nabla f(x_0, y_0)$ também é perpendicular (ortogonal) à reta tangente à curva de nível $f(x, y) = k$
                que passa por $P = (x_0, y_0)$.

            \subsubsection{Em $\mathbbm{R}^3$}
                \paragraph{}
                O vetor gradiente $\nabla F(x_0, y_0, z_0)$ é perpendicular ao plano tangente à superfície $F(x, y, z) = k$ que passa
                por $P = (x_0, y_0, z_0)$.

            \subsubsection{Plano Tangente}
                \paragraph{}
                Dado $f$, o plano tangente à $f$ em $(x_0, y_0, z_0)$ é:

                \begin{align*}
                    z - z_0 &= \PartialDer{f}{x}(x_0, y_0) (x - x_0) + \PartialDer{f}{y}(x_0, y_0) (y - y_0)\\
                    \nabla F(x_0, y_0, z_0) \cdot (x - x_0, y - y_0, z - z_0) &= 0\\
                \end{align*}

            \subsubsection{Reta Normal}
                \paragraph{}
                O vetor gradiente é múltiplo da reta normal, portanto temos o ponto $P$ e o vetor diretor da reta como
                $\nabla F(x_0, y_0, z_0)$:

                \begin{align*}
                    \nabla F (x_0, y_0, z_0) &= \lambda (x - x_0, y - y_0, z - z_0) \text{ , sendo }\lambda \in \mathbbm{R}\\
                \end{align*}

    \section{Exemplos}
        \subsection{Exemplo 6}
            \paragraph{}
            Determine a derivada direcional sendo $u$ o vetor unitário dado pelo ângulo $\theta = \pi/6$ no ponto $(1, 2)$

            \begin{align*}
                f(x, y) &= x^3 - 3xy + 4y^2\\
                D_u f(1, 2) &= \nabla f(1, 2) \cdot u\\
                \nabla f &= (3x^2 - 3y, -3x + 8y)\\
                \nabla f(1, 2) &= (-3, 13)\\
                \text{Determinando o vetor u...}\\
                u &= (\cos{\pi/6}, \sin{\pi/6}) = (\frac{\sqrt{3}}{2}, \frac{1}{2})\\
                \text{Portanto, a derivada direcional...}\\
                D_u f(1, 2) &= \frac{1}{2} (13 - 3\sqrt{3})
            \end{align*}

        \subsection{Exemplo 7}
            \paragraph{}
            Determine a derivada direcional da função no ponto $P = (2, -1)$ na direção $v = 2i + 5j$

            \begin{align*}
                f(x, y) &= x^2y^3 - 4y\\
                \nabla f &= (2xy, 3x^2y^2 - 4)\\
                \nabla f(2, -1) &= (-4, 8)\\
                D_u f(2, -1) &= \nabla f(2, -1) \cdot u\\
                \text{$\norm{v} \neq 1$, portanto precisamos encontrar u}\\
                u &= \frac{v}{\norm{v}} = \frac{2i + 5i} {\sqrt{4 + 25}} = \left(\frac{2}{\sqrt{24}}, \frac{5}{\sqrt{24}}\right)\\
                D_u f(2, -1) &= \frac{32}{\sqrt{29}}
            \end{align*}

        \subsection{Exemplo 10}
            \paragraph{}
            Determine o plano tangente e a reta normal no ponto $(-2, 1, -3)$ ao elipsóide.

            \begin{align*}
                x^2 / 4 + y^2 + z^2 / 9 &= 3\\
                F(x, y, z) &= \frac{x^2}{4} + y^2 + \frac{z^2}{9} - 3
                \nabla F(x, y, z) &= \left( \frac{x}{2}, 2y, \frac{2y}{9} \right)\\
                \nabla F(-2, 1, -3) &= (-1, 2, -2/3)\\
                \text{Portanto agora podemos determinar o plano tangente}\\
                \nabla F(-2, 1, -3) \cdot (x + 2, y - 1, z + 3) &= 0\\
                -(x + 2) + 2(y-1) - \frac{2}{3}(z + 3) &= 0\\
                \text{E a equação da reta normal é dada por}\\
                (x + 2, y - 1, z + 3) &= \lambda (-1, 2, -2/3)\\
            \end{align*}

    \begin{appendix}
        \section{Direção e vetor unitário}
            \paragraph{}
            Caso seja pedido a direção, pode ser $v$ com $\norm{v} \neq 1$, porém para os cálculos é necessário um vetor unitário $u$ e se
            $u$ estiver no mesmo sentido de $v$, então $u = \frac{v}{\norm{v}}$. Portanto é importante sempre checar se o vetor dado é unitário
            ou não.
    \end{appendix}
\end{document}
