\documentclass{article}

\usepackage[utf8]{inputenc}
\usepackage[T1]{fontenc}
\usepackage[portuguese]{babel}
\usepackage{amsmath, amsthm, amssymb}
\usepackage{graphicx}
\usepackage{bbm}

\author{Esdras R. Carmo - 170656}
\title{Aplicações de Integrais Duplas}
\date{\today}

% Comando para integral dupla sobre região R
\newcommand{\doubleint}[1] {\iint\limits_R #1 dA}
\newcommand{\doubleintp}[1] {\iint\limits_{R_{r\theta}} #1 r dr d\theta}

\begin{document}
    \maketitle

    \section{Cálculo de Massa e Carga}
        \subsection{Massa}
            Para o cálculo de massa de uma lâmina que ocupa a região $R$, tendo $\rho(x,y)$, usamos:

            \[
                m = \doubleint{\rho(x,y)}
            \]

        \subsection{Carga Elétrica}
            A carga elétrica total funciona da mesma maneira, sendo $\sigma(x,y)$ a densidade de carga:

            \[
                Q = \doubleint{\sigma(x,y)}
            \]

    \section{Momentos e Centro de Massa}
        \subsection{Momentos}
            Definimos os momentos como $M_x$ o momento em relação ao eixo $x$ e $M_y$ com relação a $y$:

            \begin{align*}
                M_x &= \doubleint{y \rho(x,y)}\\
                M_y &= \doubleint{x \rho(x,y)}\\
            \end{align*}
        
        \subsection{Centro de Massa}
            Temos as coordenadas do centro de massa $(\bar{x}, \bar{y})$, onde $m$ é a massa do corpo:

            \begin{align*}
                \bar{x} &= \frac{M_y}{m} = \frac{1}{m} \doubleint{x \rho(x,y)}\\
                \bar{y} &= \frac{M_x}{m} = \frac{1}{m} \doubleint{y \rho(x,y)}\\
            \end{align*}
\end{document}
