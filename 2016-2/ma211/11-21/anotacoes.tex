\documentclass{article}

\usepackage[utf8]{inputenc}
\usepackage[T1]{fontenc}
\usepackage[portuguese]{babel}
\usepackage{amsmath, amsthm, amssymb}
\usepackage{graphicx}
\usepackage{bbm}

\author{Esdras R. Carmo - 170656}
\title{Teorema Fundamental das Integrais de Linhas}
\date{\today}

\newcommand{\REAL}{\mathbbm{R}}

% Comando para integrais duplas e triplas
\newcommand{\doubleint}[2] {\iint\limits_{#1} #2}
\newcommand{\tripleint}[2] {\iiint\limits_{#1} #2}

% Comando para a norma
\newcommand{\norm}[1] {\left.\parallel #1 \right.\parallel}
\newcommand{\PartialDer}[2] {\frac{\partial #1}{\partial #2}}

% theorems
\newtheorem{theorem}{Teorema}[section]
\newtheorem{example}{Exemplo}[section]
\newtheorem{definition}{Definição}[section]


\begin{document}
    \maketitle

    \section{Teorema para Integrais de Linha}
        Seja $C$ uma curva lisa descrita por $r(t)$, $a \leq t \leq b$. Se $f$ uma função
        diferenciável de duas ou três variáveis cujo vetor gradiente $\nabla f$ é contínuo em
        $C$. Então:
        \[
            \int_C \nabla f \cdot dr = f(r(b)) - f(r(a))
        \]

        Análogo ao teorema fundamental do cálculo, sendo que o gradiente faz o papel da derivada.

        \subsection{Campo vetorial conservativo}
            Se o campo vetorial for gradiente de uma função, então é dito campo conservativo. Portanto,
            para esse caso, podemos calcular a integral conhecendo apenas o valor de $f$ nas extremidades
            da curva $C$, se considerarmos o campo $F$:
            \[
                F = \nabla f
            \]

        \subsection{Exemplo}
            \begin{example}
                Determine o trabalho realizado pelo campo gravitacional:
                \[
                    F(x) = - \frac{mMG}{\norm{x}^3} x
                \]

                ao mover uma particula de massa $m$ do ponto $(3, 4, 12)$ para o ponto $(2, 2, 0)$ ao longo
                de uma linha reta.

                Note que o trabalho é dado por:
                \[
                    W = \int_C F \cdot dr
                \]

                Sabemos que o campo gravitacional é conservativo, i. e.:
                \[
                    F = \nabla f
                \]

                onde:
                \[
                    f(x, y, z) = \frac{mMG}{\sqrt{x^2 + y^2 + z^2}}
                \]

                Portanto, temos a seguinte integral, que pode ser resolvida pelo teorema fundamental das integrais de linha:
                \begin{align*}
                    W &= \int_C \nabla f \cdot dr\\
                    W &= f(2, 2, 0) - f(3, 4, 12)\\
                    W &= \frac{mMG}{\sqrt{8}} - \frac{mMG}{13}\\
                    W &= mMG \left(\frac{1}{2\sqrt{2}} - \frac{1}{13}\right)
                \end{align*}
            \end{example}

    \section{Independência do Caminho}
        Dizemos que a integral de linha é independente do caminho se a integral sobre $C_1$ é igual à integral sobre
        $C_2$ para quaisquer $C_1$ e $C_2$ que possuam os mesmos pontos iniciais e finais.

        Note que a integral de linha de um campo vetorial conservativo $F = \nabla f$ é independente do caminho, pois
        pelo teorema fundamental das integrais de linha, essa integral só irá depender do ponto inicial e final aplicados em $f$.

    \section{Curva Fechada}
        Uma curva é fechada se o ponto final e inicial são equivalentes, i. e., $r(b) = r(a)$.

        \begin{theorem}
            A integral de linha $\int_C F \cdot dr$ é independente do caminho em $D$ se e somente se $\int_C F \cdot dr = 0$
            para toda curva fechada $C$.
        \end{theorem}

    \section{Determinando se $F$ é um campo vetorial conservativo}
        \label{sec:conservativo}
        Teoricamente, se a integral de linha é $0$ para qualquer curva fechada $C$, então $F$ é um campo vetorial conservativo.
        No entanto, esse teorema não nos ajuda na prática.

        Supondo que
        \[
            F(x,y) = P(x,y) i + Q(x,y) j
        \]
        é um campo vetorial conservativo, então existe $f$ tal que $F = \nabla f$, portanto,
        \begin{align*}
            P &= \PartialDer{f}{x}\\
            Q &= \PartialDer{f}{y}
        \end{align*}

        Pelo teorema de Clairaut, então:
        \[
            \PartialDer{P}{y} = \PartialDer{Q}{x}
        \]

        Assim, caso essa igualdade não for válida, então o campo vetorial não é conservativo.

    \section{Tipos de Região}
        \subsection{Curva Simples}
            Uma curva é dita simples se ela não se autointercepta em nenhum ponto entre as extremidades, excluindo os extremos.
            Podemos ter uma curva simples e fechada, por exemplo.

        \subsection{Região Simplesmente Conexa}
            Uma região $D$ é simplesmente conexa se toda curva simples fechada em $D$ contorna somente pontos que estão em
            $D$. Em outras palavras, a região não pode ter buracos. E também não é constituída por pedaços separados.

    \section{Recíproca do Teorema da Seção~\ref{sec:conservativo}}
        \label{sec:reciproca}
        Se $F = P i + Q j$ um campo vetorial sobre uma região $D$ aberta e simplesmente conexa. Se $P$ e $Q$ possuem derivadas
        parciais de primeira ordem contínuas e
        \[
            \PartialDer{P}{y} = \PartialDer{Q}{x}
        \]
        então $F$ é um campo vetorial conservativo.

    \section{Exemplos}
        \begin{example}
            Determine se o campo vetorial
            \[
                F(x,y) = (x - y)i + (x - 2)j
            \]
            é ou não conservativo.

            Como não especifica a região $D$, então está considerando todo o plano $xy$. Portanto, é uma região simplesmente
            conexa e podemos aplicar a recíproca apresentada na Seção~\ref{sec:reciproca}.

            Considerando:
            \begin{align*}
                P(x,y) &= x - y\\
                Q(x,y) &= x - 2
            \end{align*}

            Temos as derivadas:
            \begin{align*}
                \PartialDer{P}{y} &= -1\\
                \PartialDer{Q}{x} &= 1\\
                \PartialDer{P}{y} &\neq \PartialDer{Q}{x}
            \end{align*}

            Portanto o campo vetorial $F$ não é conservativo.
        \end{example}

        \begin{example}
            \label{ex:conservativo}
            Determine se o campo vetorial
            \[
                F(x,y) = (3 + 2xy) i + (x^2 - 3y^2) j
            \]
            é ou não conservativo.

            Consideremos
            \begin{align*}
                P(x,y) &= 3 + 2xy\\
                Q(x,y) &= x^2 - 3y^2
            \end{align*}

            Assim, temos as derivadas:
            \begin{align*}
                \PartialDer{P}{y} &= 2x\\
                \PartialDer{Q}{x} &= 2x\\
                \PartialDer{P}{y} &= \PartialDer{Q}{x}
            \end{align*}

            Logo, pela Seção~\ref{sec:reciproca}, o campo vetorial $F$ é conservativo.
        \end{example}

        \begin{example}
            \textbf{a)} Se $F(x,y) = (3 + 2xy) i + (x^2 - 3y^2) j$, determine uma função $f$ tal que
            $F = \nabla f$.
        
            Como já sabemos que $F$ é um campo vetorial conservativo pelo Exemplo~\ref{ex:conservativo}, então
            temos:
            \begin{align*}
                \nabla f &= \PartialDer{f}{x}i + \PartialDer{f}{y}j = P(x,y) i + Q(x,y) j\\
                \PartialDer{f}{x} &= 3 + 2xy\\
                f(x,y) &= \int (3 + 2xy) dx = 3x + x^2y + g(y)\\
                \PartialDer{f}{y} &= x^2 + g^\prime (y) = x^2 - 3y^2\\
                g^\prime (y) &= -3y^2\\
                g(y) &= - \int 3y^2 dy = -y^3 + K\\
            \end{align*}

            Logo, a função que estávamos procurando:
            \[
                f(x,y) = 3x + x^2y -y^3 + K
            \]
            onde $K$ é uma constante real.

            \textbf{b)} Calcule a integral de linha $\int_C F \cdot dr$, em que $C$ é a curva dada por
            \[
                r(t) = e^t \sin t \textbf{i} + e^t \cos t \textbf{j}
            \]
            com $0 \leq t \leq \pi$.

            Pelo teorema fundamental das integrais de linha, então
            \[
                \int_C F \cdot dr = f(r(\pi)) - f(r(0))
            \]

            Logo:
            \begin{align*}
                \int_C F \cdot dr &= f(0, -e^\pi) - f(0, 1)\\
                &= -\left(-e^\pi\right)^3 + K - (-1^3 + K)\\
                \int_C F \cdot dr &= e^{3\pi} + 1
            \end{align*}
        \end{example}

        \begin{example}
            Se
            \[
                F(x, y, z) = y^2 \textbf{i} + (2xy + e^{3z}) \textbf{j} + 3ye^{3z} \textbf{k}
            \]
            é um campo vetorial conservativo, determine $f$ função potencial.

            Temos as seguintes relações entre $f$ e $F$:
            \begin{align*}
                \PartialDer{f}{x} &= y^2\\
                \PartialDer{f}{y} &= 2xy + e^{3z}\\
                \PartialDer{f}{z} &= 3ye^{3z}
            \end{align*}

            Temos que:
            \begin{align*}
                f(x, y, z) &= \int y^2 dx = xy^2 + g(y, z)\\
                \PartialDer{f}{y} &= 2xy + g_y (y, z) = 2xy + e^{3z}\\
                g (y, z) &= \int e^{3z} dy = ye^{3z} + h(z)\\
                f(x, y, z) &= xy^2 + ye^{3z} + h(z)\\
                \PartialDer{f}{z} &= 3ye^{3z} + h^\prime(z) = 3ye^{3z}\\
                h^\prime(z) &= 0\\
                h(z) &= K \in \REAL\\
                f(x, y, z) &= xy^2 + ye^{3z} + K\\
            \end{align*}
        \end{example}
\end{document}
