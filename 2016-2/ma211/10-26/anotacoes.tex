\documentclass{article}

\usepackage[utf8]{inputenc}
\usepackage[T1]{fontenc}
\usepackage[portuguese]{babel}
\usepackage{amsmath, amsthm, amssymb}
\usepackage{graphicx}
\usepackage{bbm}

\author{Esdras R. Carmo - 170656}
\title{Integrais Triplas em Coordenadas Esféricas}
\date{\today}

\newcommand{\REAL}{\mathbbm{R}}

% Comando para integrais duplas e triplas
\newcommand{\doubleint}[2] {\iint\limits_{#1} #2}
\newcommand{\tripleint}[2] {\iiint\limits_{#1} #2}

% theorems
\newtheorem{theorem}{Teorema}[section]
\newtheorem{example}{Exemplo}[section]
\newtheorem{definition}{Definição}[section]


\begin{document}
    \maketitle

    \section{Mudança de Coordenadas}
        \paragraph{}
        Dado um ponto $P$ com coordenadas $(x, y, z)$ no sistema cartesiano, ele pode ser representado
        da seguinte forma em coordenadas esféricas $(\rho, \theta, \phi)$:

        \begin{align*}
            \rho &= \sqrt{x^2 + y^2 + z^2}\\
            \theta &= \arctan \frac{y}{x}\\
            z &= \rho \cos \phi\\
            x &= \rho \sin \phi \cos \theta\\
            y &= \rho \sin \phi \sin \theta\\
        \end{align*}

        Onde $\rho \geq 0$ e $0 \leq \phi \leq \pi$.


        \subsection{Exemplo de Sólidos}
            \subsubsection{Cunha Esférica}
                Uma cunha esférica é dada por:
                \[
                    E = \{ (\rho, \theta, \phi) \mid a \leq \rho \leq b, \alpha \leq \pi, c \leq \phi \leq d \}
                \]
                em que $0 \leq a, \beta - \alpha \leq 2\pi, 0 \leq c$ e $d \leq \pi$.

                Olhando esse sólido em um sistema de coordenadas esféricas daria uma caixa, e no sistema cartesiano
                justamente uma cunha esférica.

        \subsection{Integral}
            Para a mudança de variável na integral temos que multiplicar o integrando por $\rho^2 \sin \phi$.

    \section{Exemplos}
        \begin{example}
            Calcule
            \[
                I = \tripleint{B} e^{(x^2 + y^2 + z^2)^{3/2}} dV
            \]
            em que B é a bola unitária
            \[
                B = \{ (x,y,z) \mid x^2 + y^2 + z^2 \leq 1 \}
            \]

            Em coordenadas esféricas temos

            \begin{align*}
                B &= \{ (\rho, \theta, \phi) \mid 0 \leq \rho \leq 1, 0 \leq \theta \leq 2\pi, 0 \leq \phi \leq \pi \}\\
                I &= \int_0^\pi \int_0^{2\pi} \int_0^1 e^{\rho^3} \rho^2 \sin \phi d\rho d\theta d\phi\\
            \end{align*}

            Tomando $u = \rho^3$, temos $du = 3 \rho^2 d\rho$. Logo,

            \begin{align*}
                I &= \int_0^\pi \int_0^{2\pi} \int_0^1 \frac{1}{3} e^u \sin\phi du d\theta d\phi\\
                I &= \frac{1}{3} \int_0^\pi \int_0^{2\pi} (e - 1) \sin\phi d\theta d\phi\\
                I &= \frac{e - 1}{3} \int_0^\pi \sin\phi d\phi\\
                I &= \frac{4}{3} \pi (e - 1)\\
            \end{align*}
        \end{example}

        \begin{example}
            Utilize coordenadas esféricas para determinar o volume do sólido delimitado abaixo pelo cone
            $z = \sqrt{x^2 + y^2}$ e acima pela esfera $x^2 + y^2 + z^2 = z$.

            Note que, temos a seguinte esfera:

            \begin{align*}
                x^2 + y^2 + z^2 - z + \frac{1}{4} &= \frac{1}{4}\\
                x^2 + y^2 + \left( z - \frac{1}{2} \right)^2 &= \left(\frac{1}{2}\right)^2
            \end{align*}

            Ou seja, uma esfera de raio $\frac{1}{2}$ centrada no ponto $\left( 0, 0, \frac{1}{2} \right)$.

            Note que o cone possui exatamente o raio da esfera para $z = \frac{1}{2}$ que é quando os dois 
            sólidos se encontram.

            Vamos descrever a esfera:

            \begin{align*}
                x^2 + y^2 + z^2 &= z\\
                \rho^2 &= \rho \cos \phi \textit{ , considerando }\rho \geq 0\\
                \rho &= \cos \phi\\
            \end{align*}

            Descrevendo o cone:

            \begin{align*}
                z &= \sqrt{x^2 + y^2}\\
                \rho \cos \phi &= \rho \sin \phi\\
                \cos \phi &= \sin \phi \Rightarrow \phi = \frac{\pi}{4}\\
            \end{align*}

            Portanto a região que temos é:
            \[
                E = \left\{ (\rho, \theta, \phi) \mid 0 \leq \rho \leq \cos \phi, 0 \leq \phi \leq \frac{\pi}{4}, 0 \leq \theta \leq 2\pi \right\}
            \]

            Então podemos calcular a integral:

            \begin{align*}
                V &= \tripleint{E}{dV}\\
                V &= \int_0^{2\pi} \int_0^{\frac{\pi}{4}} \int_0^{\cos \phi} \rho^2 \sin \phi d\rho d\phi d\theta\\
                V &= \frac{1}{3} \int_0^{2\pi} \int_0^{\frac{\pi}{4}} \cos^3 \phi \sin \phi d\phi d\theta\\
            \end{align*}

            Tomando $u = \cos \phi$, temos $du = -\sin\phi d\phi$. Logo,

            \begin{align*}
                V &= -\frac{1}{3} \int_0^{2\pi} \int_0^{\frac{\sqrt{2}}{2}} u^3 du d\theta\\
                V &= -\frac{1}{12} \int_0^{2\pi} \left( \frac{1}{4} - 1 \right) d\theta\\
                V &= \frac{\pi}{8}
            \end{align*}
        \end{example}
\end{document}
