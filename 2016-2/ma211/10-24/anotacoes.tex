\documentclass{article}

\usepackage[utf8]{inputenc}
\usepackage[T1]{fontenc}
\usepackage[portuguese]{babel}
\usepackage{amsmath, amsthm, amssymb}
\usepackage{graphicx}
\usepackage{bbm}

\author{Esdras R. Carmo - 170656}
\title{Integrais Triplas em Coordenadas Cilíndricas}
\date{\today}

\newcommand{\REAL}{\mathbbm{R}}

% Comando para integrais duplas e triplas
\newcommand{\doubleint}[2] {\iint\limits_{#1} #2}
\newcommand{\tripleint}[2] {\iiint\limits_{#1} #2}

% theorems
\newtheorem{theorem}{Teorema}[section]
\newtheorem{example}{Exemplo}[section]
\newtheorem{definition}{Definição}[section]


\begin{document}
    \maketitle

    \section{Mudança de Coordenadas}
        Temos a seguinte relação:

        \begin{align*}
            x &= r \cos\theta\\
            y &= r \sin\theta\\
            z &= z\\
            r^2 &= x^2 + y^2\\
        \end{align*}

        Ou seja, temos $x$ e $y$ em coordenadas polares e $z$ livre em cartesiana. Essas coordenadas são
        boas para descrever figuras de revolução.

        \subsection{Exemplos}
            \begin{example}
                Descreva a superfície cuja equação em coordenadas cilindricas é
                $z = r$.

                Temos:

                \begin{align*}
                    z^2 &= r^2\\
                    z^2 &= x^2 + y^2
                \end{align*}

                Ou seja, temos um cone.
            \end{example}

        \subsection{Integrais Triplas}
            Se precisamos calcular a integral $\tripleint{E}{f(x,y,z) dV}$ e 
            \[
                E = \{(x,y,z) \in \REAL^3 \mid (x,y) \in D, u_1(x,y) \leq z \leq u_2(x,y) \}
            \]
            
            E além disso:
            \[
                D = \{(r, \theta) \mid \alpha \leq \theta \leq \beta, h_1(\theta) \leq r \leq h_2(\theta) \}
            \]

            Então:

            \begin{align*}
                \tripleint{E}{f(x,y,z) dV} &= \int_\alpha^\beta \int_{h_1(\alpha)}^{h_2(\beta)} \int_{u_1(r,\theta)}^{u_2(r,\theta)} f(r, \theta, z) r dz dr d\theta
            \end{align*}

    \section{Exemplos}
        \begin{example}
            Um sólido $E$ está contido no cilindro $x^2 + y^2 = 1$, abaixo do plano $z = 4$ e acima do paraboloide
            $z = 1 - x^2 - y^2$. A densidade em qualquer ponto é proporcional à distância do ponto ao eixo do cilindro. Determine
            a massa de E.

            Lembramos que a massa é:
            \[
                m = \tripleint{E}{\rho(x,y,z) dV}
            \]

            Descrevendo o sólido $E$:

            \begin{align*}
                E &= \{ (x,y,z) \in \REAL^3 \mid -1 \leq x \leq 1, -\sqrt{1 - x^2} \leq y \leq \sqrt{1 - x^2}, 1 - x^2 - y^2 \leq z \leq 4\}\\
                E_{r\theta z} &= \{ (r, \theta, z) \mid 0 \leq r \leq 1, 0 \leq \theta \leq 2\pi, 1 - r^2 \leq z \leq 4 \}
            \end{align*}

            Agora podemos calcular a integral, sabendo que $\rho(r, \theta, z) = kr$:

            \begin{align*}
                m &= \tripleint{E_{r\theta z}}{\rho(x,y,z) r dz dr d\theta}\\
                m &= \int_0^{2\pi} \int_0^1 \int_{1 - r^2}^{4} kr^2 dz dr d\theta\\
                m &= k \int_0^{2\pi} \int_0^1 r^2 (3 + r^2) dr d\theta\\
                m &= k \int_0^{2\pi} \int_0^1 (3r^2 + r^4) dr d\theta\\
                m &= k \int_0^{2\pi} \left(1 + \frac{1}{5}\right) d\theta\\
                m &= \frac{6}{5}k \int_0^{2\pi} d\theta\\
                m &= \frac{12 k\pi}{5}
            \end{align*}
        \end{example}

        \begin{example}
            Calcule
            \[
                \int_{-2}^2 \int_{-\sqrt{4 - x^2}}^{\sqrt{4 - x^2}} \int_{\sqrt{x^2 + y^2}}^2 (x^2 + y^2) dz dy dx
            \]

            Note que a integral é muito mais fácil de calcular em coordenadas cilíndricas. 
            De fato, podemos calcular da seguinte forma:

            \begin{align*}
                I &= \int_0^{2\pi} \int_0^2 \int_r^2 r^2 r dz dr d\theta\\
                I &= \int_0^{2\pi} \int_0^2 r^3 [z]_r^2 dr d\theta\\
                I &= \int_0^{2\pi} \int_0^2 (2r^3 - r^4) dr d\theta\\
                I &= \int_0^{2\pi} \left[ \frac{r^4}{2} - \frac{r^5}{5} \right]_0^2 d\theta\\
                I &= \int_0^{2\pi} \left( 8 - \frac{32}{5} \right) d\theta\\
                I &= \left[\frac{8}{5} \theta \right]_0^{2\pi}\\
                I &= \frac{16\pi}{5}
            \end{align*}
        \end{example}
\end{document}
