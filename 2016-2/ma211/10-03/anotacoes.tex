\documentclass{article}

\usepackage[utf8]{inputenc}
\usepackage[T1]{fontenc}
\usepackage[portuguese]{babel}
\usepackage{amsmath}
\usepackage{amsthm}
\usepackage{bbm}

\author{Esdras R. Carmo - 170656}
\title{Integrais Duplas sobre Retângulos}
\date{\today}

% Comando para integral dupla sobre retângulo
\newcommand{\doubleint}[1] {\iint\limits_R #1 dA}

% Definição para os exemplos
\theoremstyle{definition}
\newtheorem{example}{Exemplo}[section]

\begin{document}
    \maketitle
   
    \section{Definição de Retângulo}
        Dado dois intervalos, podemos definir o retângulo $R$ como o produto cartesiano
        entre eles, ou seja, todas as possibilidades de $(x, y) \in \mathbbm{R}^2$
        que satisfazem ambos os intervalos:

        \begin{align*}
            R = [a, b] \times [c, d] = \{(x, y) \in \mathbbm{R}^3 \mid a \leq x \leq b, c \leq y \leq d\}
        \end{align*}

    \section{Volume de $f$ definido em um retângulo}
        Dado uma função $f: \mathbbm{R}^2 \longrightarrow \mathbbm{R}$, podemos aproximar
        o volume dela em $R$ como:

        \begin{align*}
            V &\approx \sum_{i=1}^n \sum_{j=1}^n f(x_i, y_j) \Delta A\\
            \Delta A &= \Delta x \cdot \Delta y\\
        \end{align*}

        Quando $n \to \infty$, temos:

        \begin{align*}
            \doubleint{f(x,y)} = \lim_{m,n \to \infty} \sum_{i=1}^m \sum_{j=1}^n f(x_i, y_j) \Delta A\\
        \end{align*}

        Chamada a integral dupla de Riemann.

        \subsection{Propriedades da Integral Dupla}
            Analogamente com o que temos para a integral simples, segue:

            \begin{align*}
                \doubleint{(f(x,y) + g(x, y))} &= \doubleint{f(x, y)} + \doubleint{g(x, y)}\\
                \doubleint{c f(x, y)} &= c \doubleint{f(x, y)}
            \end{align*}
            E ainda temos um comportamento linear, pois se $f(x, y) \geq g(x, y)$ $\forall$ $x, y \in \mathbbm{R}$, então:

            \begin{align*}
                \doubleint{f(x,y)} \geq \doubleint{g(x,y)}
            \end{align*}

        \subsection{Valor Médio de $f$}
            O valor médio da função $f$ no retângulo $A$ é dado por:

            \begin{align*}
                \bar{f} &= \frac{1}{A(R)} \doubleint{f(x,y)}
            \end{align*}
            Em que $A(R)$ é a área do retângulo $R$.
            
            \paragraph{}
            Além disso, se $f(x,y) > 0$, então:
            
            \begin{align*}
                A(R) \cdot \bar{f} &= \doubleint{f(x,y)}
            \end{align*}

        \subsection{Teorema de Fubini}
            Por Teorema de Fubini, podemos definir nossa integral dupla sobre retângulo
            da seguinte forma:

            \begin{align*}
                R &= [a,b] \times [c, d]\\
                V &= \int_a^b A(x) dx\\
                V &= \int_a^b \int_c^d f(x,y) dy dx\\
            \end{align*}

        \subsection{Caso especial para função produto}
            Caso temos uma função $f(x,y) = g(x) \cdot h(y)$, então podemos:

            \begin{align*}
                R &= [a,b] \times [c,d]\\
                \doubleint{f(x,y)} &= \int_a^b \left( \int_c^d g(x) h(y) dy \right) dx\\
                \doubleint{f(x,y)} &= \int_a^b g(x) dx \cdot \int_c^d h(y) dy
            \end{align*}


    \section{Exemplos}
        \begin{example}
            Calcule a integral dupla:

            \begin{align*}
                \doubleint{(x - 3y^2)}
            \end{align*}

            Em que:

            \begin{align*}
                R = \{ (x,y) \in \mathbbm{R} \mid 0 \leq x \leq 2, 1 \leq y \leq 2\}
            \end{align*}

            Separando a integral e calculando a integral iterada, temos:

            \begin{align*}
                \doubleint{(x - 3y^2)} &= \int_0^2 \int_1^2 (x - 3y^2) dy dx\\
                &= \int_0^2 \left[ xy - y^3 \right]_1^2 dx\\
                &= \int_0^2 (x - 7) dx\\
                &= \left[ \frac{x^2}{2} - 7x \right]_0^2\\
                \doubleint{(x - 3y^2)} &= -12\\
            \end{align*}
        \end{example}

        \begin{example}
            Calcule a integral dupla

            \begin{align*}
                \doubleint{y\sin{(xy)}}
            \end{align*}
            em que
            \begin{align*}
                R = [1,2] \times [0, \pi]
            \end{align*}

            Note que se iniciarmos a integrar por $dy$, teremos um processo muito
            mais trabalhoso, portanto faremos:

            \begin{align*}
                \doubleint{y\sin{(xy)}} &= \int_0^\pi \int_1^2 y \sin{(xy)} dx dy\\
                &= \int_0^\pi \left[-\cos{(xy)}\right]_1^2 dy\\
                &= - \int_0^\pi (\cos{(2y)} \cos{y}) dy\\
                &= - \left[ \frac{\sin{(2y)}}{2} - \sin{y} \right]_0^\pi\\
                \doubleint{y\sin{(xy)}} &= 0
            \end{align*}
        \end{example}

        \begin{example}
            Determine o volume do sólido $S$ que é delimitado pelo paraboloide
            elíptico:

            \begin{align*}
                x^2 + 2y^2 + z &= 16
            \end{align*}

            e pelos planos $x = 2$, $y = 2$ e pelos três planos coordenados.

            \paragraph{}
            Note que a região de integração é o mesmo que $R = [0,2] \times [0,2]$.
            Além disso, temos $z = f(x,y)$ onde $f(x,y) = 16 - x^2 - 2y^2$. Portanto:

            \begin{align*}
                V &= \doubleint{f(x,y)}\\
                V &= \int_0^2 \int_0^2 [16 - x^2 - 2y^2] dx dy\\
                V &= \int_0^2 \left[ 16x - \frac{x^3}{3} - 2y^2x \right]_0^2 dy\\
                V &= \int_0^2 \left[ 32 - \frac{8}{3} - 4y^2 \right] dy\\
                V &= \left[ 32y - \frac{8y}{3} - \frac{4}{3} y^3 \right]_0^2\\
                V &= 48
            \end{align*}
        \end{example}
\end{document}
