\documentclass{article}

\usepackage[utf8]{inputenc}
\usepackage[T1]{fontenc}
\usepackage[portuguese]{babel}
\usepackage{amsmath}
\usepackage{bbm}

\author{Esdras R. Carmo - 170656}
\title{Transformações Lineares}
\date{\today}

% Comando para conjuntos numéricos
\newcommand{\REAL} {\mathbbm{R}}
\newcommand{\COMPLEX} {\mathbbm{C}}

% Comando para a norma
\newcommand{\norm}[1] {\left.\parallel #1 \right.\parallel}

\begin{document}
    \maketitle
    \paragraph{}
    As transformações lineares podem ser vistas como função que relaciona dois espaços vetoriais,
    um chamado de domínio e outro de contra-domínio, da seguinte forma: $T: V \longrightarrow W$.

    \section{Axiomas da linearidade}
        \begin{enumerate}
            \item $T(v + w) = T(v) + T(w)$
            \item $T(\lambda v) = \lambda T(v)$
        \end{enumerate}

    \section{Transformações Lineares como matrizes}
        \paragraph{}
        Sendo $T: V \longrightarrow W$ uma transformação linear, temos, se $w \in V$ e $E_n \in W$:

        \begin{align*}
            w &= \alpha_1 v_1 + \alpha_2 v_2 + \dots + \alpha_n v_n\\
            T(w) &= T(\alpha_1 v_1 + \alpha_2 v_2 + \dots + \alpha_n v_n)\\
            T(w) &= \alpha_1 T(v_1) + \alpha_2 T(v_2) + \dots + \alpha_n T(v_n)\\
            T(v_n) &= E_n\\
        \end{align*}

        \paragraph{}
        Dessa forma podemos representar as transformações lineares como multiplicação matricial.
        Se $T: \REAL^n \longrightarrow \REAL^m$, $X \in \REAL^n$, então $T(X) = AX$ sendo $A$ uma
        matriz $m \times n$.

        \subsection{Kernel (Núcleo)}
            \paragraph{}
            Sendo $A$ a matriz da transformação linear $T$, então $\text{Ker } A$ é um subespaço vetorial.

            \begin{align*}
                \text{Ker } A = \{x \in V | T(x) = 0\}\\
            \end{align*}

        \subsection{Imagem}
            \paragraph{}
            A  imagem Im $A$ é um subespaço vetorial de W tal que: 

            \begin{align*}
                \text{Im } A = \{ y \in W | \exists x \in V, y = T(x) \}\\
            \end{align*}

    \section{Teorema Posto-nulidade}
        \paragraph{}
        Sendo Nulidade a dimensão do kernel de $A$ e posto a dimensão da imagem de $A$, então:
        \begin{align*}
            \text{Nul } A + \text{ Posto } A &= \text{ dim } V\\
        \end{align*}

        \paragraph{}
        Escalonando a matriz $A$, podemos dizer que a quantidade de linhas com zeros é a nulidade de $A$, portanto, a partir desse
        teorema é possível identificar seu posto.

    \section{Posto-linha e Posto-coluna}
        \begin{align*}
            \text{Posto } A &= \text{ Posto } A^t\\
        \end{align*}

        \paragraph{}
        Dessa forma, afirmamos que o posto-linha da matriz $A$ é igual ao seu posto-coluna, ou seja, há a mesma quantidade de vetores
        L.I. em $A$ e $A^t$.
\end{document}
