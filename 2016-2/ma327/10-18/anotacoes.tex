\documentclass{article}

\usepackage[utf8]{inputenc}
\usepackage[T1]{fontenc}
\usepackage[portuguese]{babel}
\usepackage{amsmath, amsthm, amssymb}
\usepackage{bbm}

\author{Esdras R. Carmo - 170656}
\title{Diagonalização}
\date{\today}

% Comando para conjuntos numéricos
\newcommand{\REAL} {\mathbbm{R}}
\newcommand{\COMPLEX} {\mathbbm{C}}
\newcommand{\POL}[1] {\mathbbm{P}_{#1} (\REAL)}

% Comando para a norma
\newcommand{\norm}[1] {\left.\parallel #1 \right.\parallel}

% Comando para a algebra
\newcommand{\kernel}[1] {\ker \left( #1 \right)}
\newcommand{\imagem}[1] {\text{Im} \left( #1 \right)}
\newcommand{\posto}[1] {\dim{\left( \imagem{#1} \right)}}
\newcommand{\nul}[1] {\dim{ \left( \kernel{#1} \right) }}
\newcommand{\deffunc}[3] {#1: #2 \longrightarrow #3}
\newcommand{\mudabase}[3] {#1_{#2 \leftarrow #3}}
\newcommand{\vetord}[2] {\begin{pmatrix}#1\\#2\end{pmatrix}} % cria matriz coluna 2 x 1
\newcommand{\vetort}[3] {\begin{pmatrix}#1\\#2\\#3\end{pmatrix}} % cria matriz coluna 3 x 1
\newcommand{\matriz}[4] {\begin{pmatrix}#1&#2\\#3&#4\end{pmatrix}}

% theorems
\newtheorem{theorem}{Teorema}[section]
\newtheorem{example}{Exemplo}[section]
\newtheorem{definition}{Definição}[section]

\renewcommand{\qedsymbol}{$\blacksquare$}

\begin{document}
    \maketitle

    \section{Definições úteis}
        \begin{theorem}
            $\deffunc{T}{V}{V}$ linear, $\dim T = n < \infty$. Então $T$ é diagonalizável $\Leftrightarrow \exists \beta$ base
            ordenada de autovetores.
        \end{theorem}

        \begin{theorem}
            (Autovetores correspondentes a autovalores distintos são sempre L.I.). Seja $\deffunc{T}{V}{V}$ linear e $(\lambda_1, v_1)$, ... ,
            $(\lambda_k, v_k)$ autopares tais que $\lambda_1, \dots, \lambda_k$ são todos distintos $\Rightarrow$ o conjunto
            $\{v_1, \dots, v_k\}$ é L.I.
        \end{theorem}

        Neste caso, se $k = \dim V$, então existe base de auto vetores, logo $T$ é diagonalizável. No entanto essa condição é somente necessária,
        não valendo a inversa.

    \section{Exemplos}
        \begin{example}
            Vamos encontrar uma base de autovetores para $A$:
            \[
                A  = \begin{pmatrix}
                        1&1&1\\
                        1&1&1\\
                        1&1&1
                     \end{pmatrix}
            \]

            Temos que, $\forall v \in \REAL^3$ $T(v) = (x + y + z) \begin{pmatrix}1\\1\\1\end{pmatrix}$ e queremos que
            \[
                (x + y + z) \begin{pmatrix}1\\1\\1\end{pmatrix} = \lambda \begin{pmatrix}x\\y\\z\end{pmatrix}
            \]

            Se $v_1 = \begin{pmatrix}1\\1\\1\end{pmatrix}$, então:
            \[
                T(v_1) = \begin{pmatrix}3\\3\\3\end{pmatrix} = 3 \begin{pmatrix}1\\1\\1\end{pmatrix} = 3 v_1
            \]

            Note que se os vetores não forem múltiplos de $v_1$, então a equação não teria solução a menos que os dois lados
            sejam 0, i. e., $\lambda = 0$.

            Portanto, adivinhando outros autovetores temos:

            \begin{align*}
                v_2 &= \begin{pmatrix}-2\\1\\1\end{pmatrix}\\
                v_3 &= \begin{pmatrix}0\\1\\-1\end{pmatrix}
            \end{align*}

            Logo $v_1, v_2, v_3$ é base de autovetores.

            
            De outra forma podemos calcular a solução do polinômio característico:

            \begin{align*}
                P(\lambda) &= \det{(A - \lambda I)} = \lambda^2 (3 - \lambda) = 0\\
                \lambda_1 &= 3\\
                \lambda_2 &= \lambda_3 = 0
            \end{align*}

            E assim podemos calcular os subespaços dos lambda:

            \begin{align*}
                V_3 &= \left\{ \vetort{1}{1}{1} \right\}\\
                V_0 &= \left\{ \vetort{1}{0}{-1}, \vetort{0}{1}{-1} \right\}\\
            \end{align*}

            Temos que os 3 autovetores encontrados são L.I.. Como $V = \REAL^3$ espaço de 
            $\dim = 3$, então temos uma base de $\REAL^3$. Sendo $\gamma$ a base ordenada com os autovetores, i. e.,

            \[
                \gamma = \left\{ \vetort{1}{1}{1}, \vetort{1}{0}{-1}, \vetort{0}{1}{-1} \right\}
            \]

            Assim podemos determinar a matriz diagonalizada como:

            \[
                \mudabase{T}{\gamma}{\gamma} = \begin{pmatrix}3&0&0\\0&0&0\\0&0&0\end{pmatrix}
            \]
        \end{example}

        \begin{example}
            \begin{align*}
                \deffunc{T}{\POL{1}}{\POL{1}}\\
                a + bx \mapsto (5a + 4b) + (-3a -2b)x
            \end{align*}
            
            Queremos autovalores, autovetores, é diagonalizável?

            Escolhemos $\beta = \{1, x\}$ canônica, então:

            \begin{align*}
                T(1) &= 5 - 3x\\
                T(x) &= 4 - 2x\\
                \mudabase{T}{\beta}{\beta} &= \matriz{5}{4}{-3}{-2}
            \end{align*}

            Assim, podemos calcular pelo polinômio característico:

            \[
                P(\lambda) = \det{(\mudabase{T}{\beta}{\beta} - \lambda I_2)} = (\lambda - 1)(\lambda - 2) = 0
            \]

            Os autovalores são distintos ($\lambda_1 = 1, \lambda_2 = 2$). Oba, então a matriz é diagonizável, já que
            $\dim{\POL{1}} = 2$ (OBA!)

            Vamos encontrar uma base de autovetores.

            Para $\lambda_1 = 1$, temos:
            \[
                V_1 = \left\{ \vetord{1}{-1} \right\} \textit{, i. e., } V_1 = \{1 - x\}
            \]

            Similarmente, para $\lambda_2 = 2$, temos:
            \[
                V_2 = \left\{ \vetord{4}{-3} \right\} \textit{, i. e., } V_2 = \{4 - 3x\}
            \]

            Portanto, temos a seguinte base de autovetores e a matriz diagonizada:

            \begin{align*}
                \gamma &= \{ 1 - x, 4 - 3x \}\\
                \mudabase{T}{\gamma}{\gamma} &= \matriz{1}{0}{0}{2}
            \end{align*}
        \end{example}
\end{document}
