\documentclass{article}

\usepackage[utf8]{inputenc}
\usepackage[T1]{fontenc}
\usepackage[portuguese]{babel}
\usepackage{amsmath, amsthm, amssymb}
\usepackage{bbm}

\author{Esdras R. Carmo - 170656}
\title{Desigualdade de Cauchy-Schwarz}
\date{\today}

% Comando para conjuntos numéricos
\newcommand{\REAL} {\mathbbm{R}}
\newcommand{\COMPLEX} {\mathbbm{C}}
\newcommand{\POL}[1] {\mathbbm{P}_{#1} (\REAL)}

% Comando para a norma
\newcommand{\norm}[1] {\left.\parallel #1 \right.\parallel}

% Comando para a algebra
\newcommand{\kernel}[1] {\ker \left( #1 \right)}
\newcommand{\imagem}[1] {\text{Im} \left( #1 \right)}
\newcommand{\posto}[1] {\dim{\left( \imagem{#1} \right)}}
\newcommand{\nul}[1] {\dim{ \left( \kernel{#1} \right) }}
\newcommand{\deffunc}[3] {#1: #2 \longrightarrow #3}
\newcommand{\mudabase}[3] {#1_{#2 \leftarrow #3}}
\newcommand{\vetord}[2] {\begin{pmatrix}#1\\#2\end{pmatrix}} % cria matriz coluna 2 x 1
\newcommand{\vetort}[3] {\begin{pmatrix}#1\\#2\\#3\end{pmatrix}} % cria matriz coluna 3 x 1
\newcommand{\matriz}[4] {\begin{pmatrix}#1&#2\\#3&#4\end{pmatrix}}
\newcommand{\interno}[0] {<\cdot, \cdot>}

% theorems
\newtheorem{theorem}{Teorema}[section]
\newtheorem{example}{Exemplo}[section]
\newtheorem{definition}{Definição}[section]

\renewcommand{\qedsymbol}{$\blacksquare$}

\begin{document}
    \maketitle

    \section{Enunciados}
        \subsection{$\REAL$-espaço}
            Se $V$ um $\REAL$-espaço com $\interno$, então $\forall u, v \in V$ temos que 
            \[
                <u,v>^2 \leq <u,u> \cdot <v,v>
            \]

            Além disso, se torna igualdade $\Leftrightarrow$ $u$ e $v$ são L. D. $\Leftrightarrow$
            $u$ e $v$ são paralelos.

        \subsection{$\COMPLEX$-espaço}
            Se $V$ um $\COMPLEX$-espaço com $\interno$, então $\forall u, v \in V$ temos que 
            \[
                \mid<u,v>\mid^2 \leq <u,u> \cdot <v,v>
            \]
    
            Temos o mesmo resultado para o espaço $\REAL$.
\end{document}
