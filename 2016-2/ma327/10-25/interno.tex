\documentclass{article}

\usepackage[utf8]{inputenc}
\usepackage[T1]{fontenc}
\usepackage[portuguese]{babel}
\usepackage{amsmath, amsthm, amssymb}
\usepackage{bbm}

\author{Esdras R. Carmo - 170656}
\title{Produto Interno}
\date{\today}

% Comando para conjuntos numéricos
\newcommand{\REAL} {\mathbbm{R}}
\newcommand{\COMPLEX} {\mathbbm{C}}
\newcommand{\POL}[1] {\mathbbm{P}_{#1} (\REAL)}

% Comando para a norma
\newcommand{\norm}[1] {\left.\parallel #1 \right.\parallel}

% Comando para a algebra
\newcommand{\kernel}[1] {\ker \left( #1 \right)}
\newcommand{\imagem}[1] {\text{Im} \left( #1 \right)}
\newcommand{\posto}[1] {\dim{\left( \imagem{#1} \right)}}
\newcommand{\nul}[1] {\dim{ \left( \kernel{#1} \right) }}
\newcommand{\deffunc}[3] {#1: #2 \longrightarrow #3}
\newcommand{\mudabase}[3] {#1_{#2 \leftarrow #3}}
\newcommand{\vetord}[2] {\begin{pmatrix}#1\\#2\end{pmatrix}} % cria matriz coluna 2 x 1
\newcommand{\vetort}[3] {\begin{pmatrix}#1\\#2\\#3\end{pmatrix}} % cria matriz coluna 3 x 1
\newcommand{\matriz}[4] {\begin{pmatrix}#1&#2\\#3&#4\end{pmatrix}}
\newcommand{\interno}[0] {<\cdot, \cdot>}

% theorems
\newtheorem{theorem}{Teorema}[section]
\newtheorem{example}{Exemplo}[section]
\newtheorem{definition}{Definição}[section]

\renewcommand{\qedsymbol}{$\blacksquare$}

\begin{document}
    \maketitle

    \section{Teoria}
        Se $\interno$ é produto interno sobre $V$ com $dim < \infty$, então:
        \[
            <u, v> = Y^t A X
        \]

        Onde:

        \begin{align*}
            X &= [u]_\beta\\
            Y &= [v]_\beta\\
            A &= (<v_j, v_i>) = (a_{ij})
        \end{align*}

    \section{Exemplo}
        Se $V = \POL{2}$ e $\beta = \{1, x, x^2\}$ definimos
        \[
            <p(x), q(x)> = \int_{-1}^1 p(x)q(x) dx
        \]
        O produto interno.

        Calculemos a matriz $A$ de $\interno$ com relação a $\beta$.

        \begin{align*}
            a_{11} &= \int_{-1}^1 v_1 v_1 dx = \int_{-1}^1 1 dx = 2\\
            a_{12} &= \int_{-1}^1 v_2 v_1 dx = \int_{-1}^1 x dx = 0 = a_{21}\\
            a_{22} &= \int_{-1}^1 v_2 v_2 dx = \int_{-1}^1 x^2 dx = \frac{2}{3}\\
            a_{23} &= \int_{-1}^1 v_3 v_2 dx = \int_{-1}^1 x^3 dx = 0 = a_{32}\\
            a_{33} &= \int_{-1}^1 v_3 v_3 dx = \int_{-1}^1 x^4 dx = \frac{2}{5}\\
            a_{13} &= \int_{-1}^1 v_3 v_1 dx = \int_{-1}^1 x^2 dx = \frac{2}{3} = a_{31}\\
        \end{align*}

        Portanto,

        \begin{align*}
            A &= 
            \begin{pmatrix}
                2 & 0 & \frac{2}{3}\\
                0 & \frac{2}{3} & 0\\
                \frac{2}{3} & 0 & \frac{2}{5}
            \end{pmatrix}
        \end{align*}

        Assim o teorema fala:

        \begin{align*}
            <a + bx + cx^2, d + ex + fx^2> &= \begin{pmatrix}d&e&f\end{pmatrix} \cdot 
            \begin{pmatrix}
                2 & 0 & \frac{2}{3}\\
                0 & \frac{2}{3} & 0\\
                \frac{2}{3} & 0 & \frac{2}{5}
            \end{pmatrix}
            \cdot
            \begin{pmatrix}
                a\\b\\c
            \end{pmatrix}
        \end{align*}
\end{document}
