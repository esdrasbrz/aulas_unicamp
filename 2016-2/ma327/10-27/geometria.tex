\documentclass{article}

\usepackage[utf8]{inputenc}
\usepackage[T1]{fontenc}
\usepackage[portuguese]{babel}
\usepackage{amsmath, amsthm, amssymb}
\usepackage{bbm}

\author{Esdras R. Carmo - 170656}
\title{Geometria com Produto Interno}
\date{\today}

% Comando para conjuntos numéricos
\newcommand{\REAL} {\mathbbm{R}}
\newcommand{\COMPLEX} {\mathbbm{C}}
\newcommand{\POL}[1] {\mathbbm{P}_{#1} (\REAL)}

% Comando para a norma
\newcommand{\norm}[1] {\left.\parallel #1 \right.\parallel}

% Comando para a algebra
\newcommand{\kernel}[1] {\ker \left( #1 \right)}
\newcommand{\imagem}[1] {\text{Im} \left( #1 \right)}
\newcommand{\posto}[1] {\dim{\left( \imagem{#1} \right)}}
\newcommand{\nul}[1] {\dim{ \left( \kernel{#1} \right) }}
\newcommand{\deffunc}[3] {#1: #2 \longrightarrow #3}
\newcommand{\mudabase}[3] {#1_{#2 \leftarrow #3}}
\newcommand{\vetord}[2] {\begin{pmatrix}#1\\#2\end{pmatrix}} % cria matriz coluna 2 x 1
\newcommand{\vetort}[3] {\begin{pmatrix}#1\\#2\\#3\end{pmatrix}} % cria matriz coluna 3 x 1
\newcommand{\matriz}[4] {\begin{pmatrix}#1&#2\\#3&#4\end{pmatrix}}
\newcommand{\interno}[0] {<\cdot, \cdot>}

% theorems
\newtheorem{theorem}{Teorema}[section]
\newtheorem{example}{Exemplo}[section]
\newtheorem{definition}{Definição}[section]

\renewcommand{\qedsymbol}{$\blacksquare$}

\begin{document}
    \maketitle

    \section{Norma e Distância}
        Se $V$ tem $\interno$ produto interno, então:

        \begin{itemize}
            \item $V$ também tem $\norm{.} = \sqrt{\interno}$ comprimento;
            \item $V$ tem uma distância $d(u,v) = \norm{v - u}$.
        \end{itemize}

        \begin{example}
            Sendo $V = \POL{2}$ com produto interno:
            \[
                <f(x), g(x)> := \int_0^1 f(x) g(x) dx
            \]

            se $f(x) = 3 + 2x$, $g(x) = x^2$.

            \begin{align*}
                <f(x), g(x)> &= \int_0^1 (3 + 2x)x^2 dx = \frac{3}{2}\\
                \norm{f(x)} &= \sqrt{<f(x), f(x)>} = \sqrt{\int_0^1 (3 + 2x)^2 dx} = \frac{9}{\sqrt{3}}
            \end{align*}
        \end{example}

    \section{Ângulo}
        Por Cauchy-Schwarz temos:

        \begin{align*}
            <u,v>^2 &\leq <u,u> <v,v> \Rightarrow \frac{<u,v>^2}{<u,u> <v,v>} \leq 1\\
            -1 &\leq \frac{<u,v>}{\sqrt{<u,u>} \sqrt{<v,v>}} \leq 1\\
            -1 &\leq \frac{<u,v>}{\norm{u} \norm{v}} \leq 1\\
        \end{align*}

        Portanto definimos:

        \begin{align*}
            \cos \theta &= \frac{<u,v>}{\norm{u} \norm{v}}
        \end{align*}

        Para $\theta \in [0, \pi]$.

    \section{Ortoganalidade}
        Se $V$ $F$-espaço com $\interno$. Os vetores $u, v \in V$ são ortogonais se, e somente se,
        $<u, v> = 0$, ou seja, $\theta = \frac{\pi}{2}$.

        \subsection{Base Ortogonal}
            Se $V$ um espaço com $\interno$ e $\dim V < \infty$ uma base $\beta$.
            
            \begin{enumerate}
                \item \textbf{Ortogonal} se $\beta$ é conjunto ortogonal, ou seja, todos os vetores de $\beta$
                    são ortogonais entre si;
                \item \textbf{Ortonormal} se $\beta$ é conjunto ortonormal, ou seja, além de ortogonais, os vetores
                    são unitários (norma = $1$).
            \end{enumerate}

            \paragraph{} Se tivermos base ortogonal, é muito simples escrever um vetor nessa base, pois teremos uma fórmula
            para dar os coeficientes da seguinte forma. Para todo $v \in V$, então:
            \[
                \alpha_i = \frac{<v, q_i>}{<q_i, q_i>}
            \]

            onde $q_i$ é o vetor $i$ da base $beta$.
\end{document}
