\documentclass{article}

\usepackage[utf8]{inputenc}
\usepackage[T1]{fontenc}
\usepackage[portuguese]{babel}
\usepackage{amsmath, amsthm, amssymb}
\usepackage{bbm}

\author{Esdras R. Carmo - 170656}
\title{Isomorfismos}
\date{\today}

% Comando para conjuntos numéricos
\newcommand{\REAL} {\mathbbm{R}}
\newcommand{\COMPLEX} {\mathbbm{C}}

% Comando para a norma
\newcommand{\norm}[1] {\left.\parallel #1 \right.\parallel}

% Comando para a algebra
\newcommand{\kernel}[1] {\ker \left( #1 \right)}
\newcommand{\imagem}[1] {\text{Im} \left( #1 \right)}
\newcommand{\posto}[1] {\dim{\left( \imagem{#1} \right)}}
\newcommand{\nul}[1] {\dim{ \left( \kernel{#1} \right) }}
\newcommand{\deffunc}[3] {#1: #2 \longrightarrow #3}

% theorems
\newtheorem{theorem}{Teorema}[section]
\newtheorem{example}{Exemplo}[section]
\newtheorem{definition}{Definição}[section]

\renewcommand{\qedsymbol}{$\blacksquare$}

\begin{document}
    \maketitle

    \section{Trasformação isomorfismo}
        \begin{definition}
            $\deffunc{T}{V}{W}$ linear é isomorfismo se $T$ é bijetora. Escrevemos
            $V \cong W$.
        \end{definition}

        \begin{theorem}
            Se $V$ é espaço vetorial tal que $\textit{dim}_F V = n < \infty$, então
            $V \cong F^n$.
        \end{theorem}
        O Teorema fala que "vetores são equivalentes a coordenadas" em algum sentido.
        \begin{proof}
            Peguem $B = {v_1, \dots, v_n}$ base ordenada de $V$.
            Agora definimos uma transformação linear $\deffunc{T}{V}{F^n}$ da seguinte
            forma: 
            
            \begin{align*}
                v \mapsto [v]_B &= \begin{pmatrix}c_1\\c_2\\\vdots\\c_n\end{pmatrix}
            \end{align*}

            Assim podemos demonstrar que $T$ é linear, sobrejetora e injetora (por construção).
        \end{proof}

        \subsection{Relação entre dimensões}
            \begin{theorem}
                Sejam $V$, $W$ espaços vetoriais com dimensão finita. Então $V \cong W \Leftrightarrow \dim{V} = \dim{W}$
            \end{theorem}
            \begin{proof}
                $\Rightarrow$ Se $V \cong W$, então $\exists \deffunc{T}{V}{W}$ isomorfismo (i. e., bijetora e linear).
                Assim $\nul{T} = 0$ e $\posto{T} = \dim W$

                \paragraph{}
                Através do teorema posto-nulidade, temos:

                \begin{align*}
                    \dim{V} = \nul{T} + \posto{T} = 0 + \dim{W} = \dim{W}
                \end{align*}

                \paragraph{}
                $\Leftarrow$ Se $\dim{V} = \dim{W} = n$, então $V \cong F^n$ e $F^n \cong W$, portanto,
                $\exists \deffunc{T}{V}{F^n}$ isomorfismo $\deffunc{S}{F^n}{W}$ isomorfismo. Logo:

                \begin{align*}
                    \deffunc{T \circ S}{V}{W} \textit{ isomorfismo}
                \end{align*}
            \end{proof}

    \section{Exemplos}
        \begin{example}
            Existe um isomorfismo entre

            \begin{align*}
                U &= \{a + bx + cx^2 + dx^3 \mid a - b = 0 = c - d\}\\
                W &= \left\{ \begin{pmatrix}x&y\\z&w\end{pmatrix} \mid x = 0 = z - 3y \right\}
            \end{align*}

            Precisamos de bases de $U$ e $W$.

            \begin{align*}
                U &\Rightarrow t + tx + sx^2 + sx^3 = t(1 + x) + s(x^2 + x^3)\\
                W &\Rightarrow \begin{pmatrix}0&t\\3t&s\end{pmatrix} =
                    t\begin{pmatrix}0&1\\3&0\end{pmatrix} + s\begin{pmatrix}0&0\\0&1\end{pmatrix}
            \end{align*}
            
            Note que os geradores são L.I., logo são bases. Portanto criamos $\deffunc{T}{U}{W}$ definindo:

            \begin{align*}
                T(1 + x) &= \begin{pmatrix}0&1\\3&0\end{pmatrix}\\
                T(x^2 + x^3) &= \begin{pmatrix}0&0\\0&1\end{pmatrix}
            \end{align*}

            Além disso existe inversa, apenas invertendo os geradores da imagem com os do domínio.
        \end{example}
\end{document}
