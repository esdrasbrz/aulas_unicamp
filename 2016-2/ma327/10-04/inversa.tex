\documentclass{article}

\usepackage[utf8]{inputenc}
\usepackage[T1]{fontenc}
\usepackage[portuguese]{babel}
\usepackage{amsmath, amsthm, amssymb}
\usepackage{bbm}

\author{Esdras R. Carmo - 170656}
\title{Transformação Inversa}
\date{\today}

% Comando para conjuntos numéricos
\newcommand{\REAL} {\mathbbm{R}}
\newcommand{\COMPLEX} {\mathbbm{C}}

% Comando para a norma
\newcommand{\norm}[1] {\left.\parallel #1 \right.\parallel}

% Comando para a algebra
\newcommand{\kernel}[1] {\ker \left( #1 \right)}
\newcommand{\imagem}[1] {\text{Im} \left( #1 \right)}
\newcommand{\posto}[1] {\dim{\left( \imagem{#1} \right)}}
\newcommand{\nul}[1] {\dim{ \left( \kernel{#1} \right) }}
\newcommand{\Span}[1] {\text{span} \left\{ #1 \right\}}
\newcommand{\deffunc}[3] {#1: #2 \longrightarrow #3}

% theorems
\newtheorem{theorem}{Teorema}[section]
\newtheorem{example}{Exemplo}[section]
\newtheorem{definition}{Definição}[section]

\renewcommand{\qedsymbol}{$\blacksquare$}

\begin{document}
    \maketitle
    
    \section{Inversas a esquerda e a direita}
        \begin{definition}
            $\deffunc{T}{V}{W}$ linear.

            \begin{itemize}
                \item Uma aplicação $\deffunc{L}{\imagem{T}}{V}$ é uma inversa
                    a esquerda se \begin{align*}(L \circ T) v = v\text{ } \forall\text{ }v \in V\end{align*}
                \item Uma aplicação $\deffunc{R}{\imagem{T}}{V}$ é inversa a direita se 
                    \begin{align*}(T \circ R) w = w\text{ }\forall\text{ }w \in \imagem{T}\end{align*}
            \end{itemize}
        \end{definition}

        \begin{theorem}
            $V$, $W$ dimensão finita (talvez diferentes). Se $\deffunc{T}{V}{W}$ tem uma inversa
            a esquerda $L$, então $T$ também tem uma inversa a direita. Além disso, $L$ é única
        \end{theorem}
        
        \begin{theorem}
            $\deffunc{T}{V}{W}$ linear. Então $T$ tem uma inversa a esquerda $\Leftrightarrow$ $T$ é injetora.
        \end{theorem}

        \begin{definition}
            Se $\deffunc{T}{V}{W}$ linear com inversa a esquerda, denotamos a únicda
            inversa a esquerda com $\deffunc{T^{-1}}{\imagem{T}}{V}$ e dizemos que $T$ é 
            invertível.
        \end{definition}

        \begin{theorem}
            Se $\deffunc{T}{V}{W}$ isomorfismo $\Rightarrow \deffunc{T^{-1}}{W}{V}$ é isomorfismo.
        \end{theorem}

    \section{Exemplos}
        \begin{example}
            Definimos $\deffunc{T}{P_1(\REAL)}{\REAL^3}$ como $a + bx \mapsto (a, 2b, a + b)$

            \begin{align*}
                \kernel{T} &= \{a + bx \mid (a, 2b, a + b) = (0, 0, 0)\} = \{(0, 0, 0)\}\\
            \end{align*}

            Dessa forma, concluimos que $T$ é injetora e possui inversa a esquerda. Agora a construímos:

            \begin{align*}
                \imagem{T} &= \Span{ (1, 0, 1), (0, 2, 1) }\\
                \deffunc{T^{-1}}{\imagem{T}}{P_1(\REAL)}\\
                (1, 0, 1) &\mapsto 1\\
                (0, 2, 1) &\mapsto x\\
                (\alpha, \beta, \gamma) &\mapsto \alpha + \frac{\beta}{2} x
            \end{align*}

            \textbf{Verificar!}
        \end{example}

        \begin{example}
            Definimos $\deffunc{T}{\REAL^3}{\REAL^2}$ como $(x, y, z) \mapsto (x - z, 2z + y)$.

            \begin{align*}
                \imagem{T} &= \Span{(1, 0), (0, 1), (-1, 2)} = \REAL^2\\
                \kernel{T} &= \Span{(1, -2, 1)} \neq \{(0, 0, 0)\}\\
            \end{align*}

            $T$ não é injetora, logo $T$ não tem inversa a esquerda. No entando, existem
            mais de uma inversa direita.

            \begin{align*}
                R_1 : \REAL^2 \to \REAL^3 \text{  } (a, b) &\mapsto (a, b, 0)\\
                R_2 : \REAL^2 \to \REAL^3 \text{  } (a, b) &\mapsto \left(a + \frac{b}{2}, 0, \frac{b}{2}\right)\\
            \end{align*}

            \textbf{Verificar!}
        \end{example}
\end{document}
