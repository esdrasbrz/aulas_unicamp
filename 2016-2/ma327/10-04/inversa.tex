\documentclass{article}

\usepackage[utf8]{inputenc}
\usepackage[T1]{fontenc}
\usepackage[portuguese]{babel}
\usepackage{amsmath, amsthm, amssymb}
\usepackage{bbm}

\author{Esdras R. Carmo - 170656}
\title{Transformação Inversa}
\date{\today}

% Comando para conjuntos numéricos
\newcommand{\REAL} {\mathbbm{R}}
\newcommand{\COMPLEX} {\mathbbm{C}}

% Comando para a norma
\newcommand{\norm}[1] {\left.\parallel #1 \right.\parallel}

% Comando para a algebra
\newcommand{\kernel}[1] {\ker \left( #1 \right)}
\newcommand{\imagem}[1] {\text{Im} \left( #1 \right)}
\newcommand{\posto}[1] {\dim{\left( \imagem{#1} \right)}}
\newcommand{\nul}[1] {\dim{ \left( \kernel{#1} \right) }}
\newcommand{\deffunc}[3] {#1: #2 \longrightarrow #3}

% theorems
\newtheorem{theorem}{Teorema}[section]
\newtheorem{example}{Exemplo}[section]
\newtheorem{definition}{Definição}[section]

\renewcommand{\qedsymbol}{$\blacksquare$}

\begin{document}
    \maketitle
    
    \section{Inversas a esquerda e a direita}
        \begin{definition}
            $\deffunc{T}{V}{W}$ linear.

            \begin{itemize}
                \item Uma aplicação $\deffunc{L}{\imagem{T}}{V}$ é uma inversa
                    a esquerda se \begin{align*}(L \circ T) v = v\text{ } \forall\text{ }v \in V\end{align*}
                \item Uma aplicação $\deffunc{R}{\imagem{T}}{V}$ é inversa a direita se 
                    \begin{align*}(T \circ R) w = w\text{ }\forall\text{ }w \in \imagem{T}\end{align*}
            \end{itemize}
        \end{definition}

        \begin{theorem}
            $V$, $W$ dimensão finita (talvez diferentes). Se $\deffunc{T}{V}{W}$ tem uma inversa
            a esquerda $L$, então $T$ também tem uma inversa a direita. Além disso, $L$ é única
        \end{theorem}
        
        \begin{theorem}
            $\deffunc{T}{V}{W}$ linear. Então $T$ tem uma inversa a esquerda $\Leftrightarrow$ $T$ é injetora.
        \end{theorem}
\end{document}
