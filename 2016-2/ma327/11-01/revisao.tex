\documentclass{article}

\usepackage[utf8]{inputenc}
\usepackage[T1]{fontenc}
\usepackage[portuguese]{babel}
\usepackage{amsmath, amsthm, amssymb}
\usepackage{bbm}
\usepackage[shortlabels]{enumitem}

\author{Esdras R. Carmo - 170656}
\title{Revisão para a P2}
\date{\today}

% Comando para conjuntos numéricos
\newcommand{\REAL} {\mathbbm{R}}
\newcommand{\COMPLEX} {\mathbbm{C}}
\newcommand{\POL}[1] {\mathbbm{P}_{#1} (\REAL)}
\newcommand{\MATRIX}[1] {\mathbbm{M}_{#1} (\REAL)}

% Comando para a norma
\newcommand{\norm}[1] {\left.\parallel #1 \right.\parallel}

% Comando para a algebra
\newcommand{\kernel}[1] {\ker \left( #1 \right)}
\newcommand{\imagem}[1] {\text{Im} \left( #1 \right)}
\newcommand{\posto}[1] {\dim{\left( \imagem{#1} \right)}}
\newcommand{\nul}[1] {\dim{ \left( \kernel{#1} \right) }}
\newcommand{\deffunc}[3] {#1: #2 \longrightarrow #3}
\newcommand{\mudabase}[3] {#1_{#2 \leftarrow #3}}
\newcommand{\vetord}[2] {\begin{pmatrix}#1\\#2\end{pmatrix}} % cria matriz coluna 2 x 1
\newcommand{\vetort}[3] {\begin{pmatrix}#1\\#2\\#3\end{pmatrix}} % cria matriz coluna 3 x 1
\newcommand{\matriz}[4] {\begin{pmatrix}#1&#2\\#3&#4\end{pmatrix}}
\newcommand{\interno}[2] {\langle #1 , #2\rangle}

% theorems
\newtheorem{theorem}{Teorema}[section]
\newtheorem{example}{Exemplo}[section]
\newtheorem{definition}{Definição}[section]

\renewcommand{\qedsymbol}{$\blacksquare$}

\begin{document}
    \maketitle

    \begin{example}
        Seja $V = \POL{2}$ com produto interno:
        \[
            \interno{p}{q} = p(-1) q(-1) + p(0) q(0) + p(1) q(1)
        \]

        \begin{enumerate}[(a)]
            \item 
                Determine a matriz $A$ com relação a $\beta = \{1, x, x^2\}$ canônica:\\
                Como $\interno{\cdot}{\cdot}$ é produto interno, então $A$ é simétrica. Então calculamos:

                \begin{align*}
                    a_{11} &= \interno{v_1}{v_1} = 3\\
                    a_{22} &= \interno{v_2}{v_2} = 2\\
                    a_{33} &= \interno{v_3}{v_3} = 2\\
                    a_{12} &= \interno{v_2}{v_1} = 0 = a_{21}\\
                    a_{13} &= \interno{v_3}{v_1} = 2 = a_{31}\\
                    a_{23} &= \interno{v_3}{v_2} = 0 = a_{32}\\
                    A &= \begin{pmatrix}
                         3 & 0 & 2\\
                         0 & 2 & 0\\
                         2 & 0 & 2
                         \end{pmatrix}
                \end{align*}

                Portanto a fórmula completa é:

                \begin{align*}
                    \interno{a + bx + cx^2}{d + ex + fx^2} &= \begin{pmatrix}d&e&f\end{pmatrix} \cdot
                                                              \begin{pmatrix}
                                                                  3 & 0 & 2\\
                                                                  0 & 2 & 0\\
                                                                  2 & 0 & 2
                                                              \end{pmatrix} \cdot
                                                              \begin{pmatrix}a\\b\\c\end{pmatrix}
                \end{align*}

            \item
                Se $p = -1 + 3x + x^2$ e $q = 4 + 2x - x^2$, encontre $\interno{p}{q}$ usando $Y^t A X$\\

                Faremos os cálculos:

                \begin{align*}
                    \interno{p}{q} &= \begin{pmatrix}4&2&-1\end{pmatrix} \cdot
                                                              \begin{pmatrix}
                                                                  3 & 0 & 2\\
                                                                  0 & 2 & 0\\
                                                                  2 & 0 & 2
                                                              \end{pmatrix} \cdot
                                                              \begin{pmatrix}-1\\3\\1\end{pmatrix}\\
                    \interno{p}{q} &= \begin{pmatrix}4&2&-1\end{pmatrix} \cdot \begin{pmatrix}-1\\6\\0\end{pmatrix} = 8
                \end{align*}

            \item
                Achem todos $t \in V$, t. q., $\interno{p}{t} = 0$, i. e., são perpendiculares.\\

                Tome $t = d + ex + fx^2$, $d, e, f \in \REAL$. Então teremos:

                \begin{align*}
                    \interno{p}{t} &= 0\\
                    \begin{pmatrix}d&e&f\end{pmatrix} \cdot \begin{pmatrix}-1\\6\\0\end{pmatrix} &= -d + 6e = 0\\
                    6e &= d\\
                    t &= 6e + ex + fx^2 \text{ }\forall\text{ } e, f \in \REAL
                \end{align*}
        \end{enumerate}
    \end{example}

    \begin{example}
        Seja $V = \MATRIX{n}$ e $\deffunc{T}{V}{V}$ tal que $A \mapsto A^t$. Mostre que os autovalores são $\lambda_1 = 1$ e $\lambda_2 = -1$.
        Encontre os autoespaços.

        Para a solução, suponha que existam autopares $(\lambda, A)$ e deduzimos $\lambda$.

        \begin{itemize}
            \item Temos, se $(\lambda, A)$ autopar:
                \[
                    T(A) = \lambda A = A^t
                \]

            \item Aplicando a transformação novamente em ambos os lados, teremos:
                \begin{align*}
                    T(T(A)) &= T(\lambda A) = \lambda T(A)\\
                    A &= \lambda^2 A.
                \end{align*}

                Como $A \neq 0$, então teremos $\lambda^2 = 1 \Rightarrow \lambda = \pm 1$

            \item Encontramos $V_1$ e $V_{-1}$.
                \begin{align*}
                    V_1 &= \{ A \in V \mid A^t = A \}\\
                    V_{-1} &= \{ A \in V \mid A^t = -A \}
                \end{align*}

                Ou seja, $V_1$ é o espaço das matrizes simétricas e $V_{-1}$ das antissimétricas.
        \end{itemize}
    \end{example}
\end{document}
