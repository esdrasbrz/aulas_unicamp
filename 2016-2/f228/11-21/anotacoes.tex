\documentclass{article}

\usepackage[utf8]{inputenc}
\usepackage[T1]{fontenc}
\usepackage[portuguese]{babel}
\usepackage{amsmath}
\usepackage{bbm}

\author{Esdras R. Carmo - 170656}
\title{Revisão básica Teoria Cinética}
\date{\today}

\begin{document}
    \maketitle

    \section{Fórmulas}
        \begin{enumerate}
        \item
            Lei dos gases:
            \begin{align*}
                p V &= nRT\\
                R &= 8,31 Jmol^{-1}K^{-1}
            \end{align*}

        \item
            Razão entre os calores específicos:
            \begin{align*}
                \gamma &= \frac{C_P}{C_V} = \frac{f + 2}{f}\\
                C_P &= C_V + R\\
                C_V &= \frac{f}{2} R
            \end{align*}

        \item
            Gases monoatômicos $f = 3$ graus de translação
        \item
            Gases diatômicos $f = 5$, com $3$ translação e $2$ rotação. As vezes pode-se
            considerar mais $2$ de vibração, então $f = 7$.
        \item
            Primeira Lei:
            \[
                Q = W + \Delta E_{int}
            \]
        \end{enumerate}

    \section{Processos Termodinâmicos básicos}
        \begin{table}[h!]
            \centering
            \caption{Tabela de Processos Termodinâmicos}

            \begin{tabular}{|l|l|l|l|l|}
                \hline
                Processo & Conceito & $\Delta E$ & $\Delta Q$ & W \\
                \hline
                Isobárico & $P \equiv cte$ & $n C_V \Delta T$ & $n C_P \Delta T$ & $P \Delta V$\\
                \hline
                Isocórico & $V \equiv cte$ & $n C_V \Delta T$ & $n C_V \Delta T$ & $0$ \\
                \hline
                Isotérmico & $T \equiv cte$ & $0$ & $n R T \ln \left( \frac{V_f}{V_i} \right)$ & $ n R T \ln \left( \frac{V_f}{V_i} \right)$ \\
                \hline
                Adiabático & $PV^\gamma \equiv cte$ & $\Delta E = -W$ & $0$ & $W = \int P dV$\\
                \hline
            \end{tabular}
        \end{table}
\end{document}
