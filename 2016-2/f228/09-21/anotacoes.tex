\documentclass{article}

\usepackage[utf8]{inputenc}
\usepackage[T1]{fontenc}
\usepackage[portuguese]{babel}
\usepackage{amsmath}
\usepackage{bbm}

\author{Esdras R. Carmo - 170656}
\title{Oscilações}
\date{\today}

% Comando para um meio
\newcommand{\meio}{\frac{1}{2}}

\begin{document}
    \maketitle
    
    \section{Variações periódicas}
        \subsection{Sistema Massa-Mola}
            \paragraph{}
            Igualando a energia mecânica pela energia potêncial e cinética e integrando, obtemos a equação periódica da
            posição:

            \begin{align*}
                x(t) &= A\sin{\omega t + \varphi_0}\\
                x(t) &= A\cos{\omega t + \phi_0}\\
                T &= \frac{1}{f}\\
                \omega &= 2\pi f = \frac{2\pi}{T}\\
                T &= \frac{2\pi}{\omega}
            \end{align*}

            \paragraph{}
            $A$, $\phi$ são dados pelas condições iniciais do seu problema (determinar a fase e amplitude), i. e.,
            $(x(t=0)$ e $v(t=0))$

            \begin{align*}
                x(t) &= A\cos{\omega t + \phi_0}\\
                v(t) = \frac{dx}{dt} &= -\omega A \sin{\omega t + \phi}\\
                a(t) = \frac{d^2x}{dt^2} &= -\omega^2 A \cos{\omega t + \phi} = -\omega^2 x(t)\\
            \end{align*}

            \paragraph{}
            Podemos ainda substituir nas equações com base nas seguintes relações:

            \begin{align*}
                v_{max} &= \omega A\\
                a_{max} &= \omega^2 A\\
            \end{align*}

            \subsubsection{Energia mecânica do sistema}
                \begin{align*}
                    E &= \meio m\omega^2 A^2 \sin^2{(\omega t + \phi)} + \meio K A^2 \cos^2{(\omega t + \phi)}\\
                \end{align*}

        \subsection{Pêndulos}
            \paragraph{}
            A partir do torque, conseguimos concluir que $\omega = \sqrt{\frac{g}{L}}$, sendo que:

            \begin{align*}
                \tau &= -r F \sin{\theta} = -Lmg\sin{\theta}\\
            \end{align*}

            \paragraph{}
            A mesma relação vale para o pêndulo físico, apenas substituindo $L$ pelo $R_{cm}$, e obtemos
            $\omega = \sqrt{\frac{R_{cm}mg}{I}}$.


\end{document} 
